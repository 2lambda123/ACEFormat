%!TEX root = ../ACEFormat.tex
\section{Continuous-Energy and Discrete Neutron Transport Tables}\label{sec:ContinuousEnergyNeutron}
The format of individual blocks found on neutron transport tables is identical for continuous-energy and discrete-reaction \ACE\ Tables; the format for both are described in this section. The blocks of data are:
\begin{enumerate}
  \item \textbf{\Block{ESZ}}---contains the main energy grid for the Table and the total, absorption, and elastic cross sections as well as the average heating numbers. The \Block{ESZ} block always exists. See \Sectionref{sec:ESZBlock}.
  \item \textbf{\Block{NU}}---contains prompt, delayed and/or total $\overline{\nu}$ as a function of incident neutron energy. The \Block{NU} exists only for fissionable isotopes; that is, if \jxs{2}$\neq0$. See \Sectionref{sec:NUBlock}.
  \item \textbf{\Block{MTR}}---contains a list of ENDF MT numbers for all neutron reactions other than elastic scattering. The \Block{MTR} exists for all isotopes that have reactions other than elastic scattering; that is, all isotopes with \nxs{4}$\neq0$. See \Sectionref{sec:MTRBlock}.
  \item \textbf{\Block{LQR}}---contains a list of kinematic $Q$-values for all neutron reactions other than elastic scattering. The \Block{LTR} exists if \nxs{4}$\neq0$. See \Sectionref{sec:LQRBlock}.
  \item \textbf{\Block{TYR}}---contains information about the type of reaction for all neutron reactions other than elastic scattering. Information for each reaction includes the number of secondary neutrons and whether secondary neutron angular distributions are in the laboratory or \CM system. The \Block{TYR} exists if \nxs{4}$\neq0$. See \Sectionref{sec:TYRBlock}.
  \item \textbf{\Block{LSIG}}---contains a list of cross section locators for all neutron reacitons other than elastic scattering. The \Block{LSIG} exists if \nxs{4}$\neq0$. See \Sectionref{sec:LSIGBlock}
  \item \textbf{\Block{SIG}}---contains cross sections for all reactions other than elastic scattering. The \Block{SIG} exists if \nxs{4}$\neq0$. See \Sectionref{sec:SIGBlock}.
  \item \textbf{\Block{LAND}}---contains a list of angular-distribution locators for all reactions producing secondary neutrons. The \Block{LAND} always exists. See \Sectionref{sec:LANDBlock}.
  \item \textbf{\Block{AND}}---contains list angular distributions for all reactions producing secondary neutrons. The \Block{AND} always exists. See \Sectionref{sec:ANDBlock}.
  \item \textbf{\Block{LDLW}}---contains a list of energy distributions for all reactions producing secondary neutrons except for elastic scattering. The \Block{LDLW} exists if \nxs{5}$\neq0$. See \Sectionref{sec:LDLWBlock}.
  \item \textbf{\Block{DLW}}---contains energy distributions for all reactions producing secondary neutrons except for elastic scattering. The \Block{DLW} exists if \nxs{5}$\neq0$. See \Sectionref{sec:DLWBlock}.
  \item \textbf{\Block{GPD}}---contains the total photon production cross section tabulated on the ESZ energy grid and a $30\times$ matrix of secondary photon energies. The \Block{GPD} exists only for those older evaluations that provide coupled neutron/photon information; that is, if \jxs{12}$\neq0$. See \Sectionref{sec:GPDBlock}.
  \item \textbf{\Block{MTRP}}---contains a list of MT numbers for all photon production reactions. The term ``photon production reaction'' is used for any information describing a specific neutron-in, photon-out reaction. The \Block{MTR} exists if \nxs{6}$\neq6$. See \Sectionref{sec:MTRBlock}.
  \item \textbf{\Block{LSIGP}}---contains a list of cross section locators for all photon production reactions. The \Block{LSIGP} exists if \nxs{6}$\neq0$. See \Sectionref{sec:LSIGBlock}.
  \item \textbf{\Block{SIGP}}---contains cross sections for all photon production reactions. The \Block{SIGP} exists if \nxs{6}$\neq0$. See \Sectionref{sec:SIGPBlock}.
  \item \textbf{\Block{LANDP}}---contains a list of angular-distribution locators for all photon production reactions. The \Block{LANDP} exist if \nxs{6}$\neq0$. See \Sectionref{sec:LANDPBlock}
  \item \textbf{\Block{ANDP}}---contains photon angular distributions for all photon production reactions. The \Block{ANDP} exists if \nxs{6}$\neq0$. See \Sectionref{sec:ANDPBlock}.
  \item \textbf{\Block{LDLWP}}---contains a list of energy-distribution locators for all photon production reactions. The \Block{LDLWP} exists if \nxs{6}$\neq0$. See \Sectionref{sec:LDLWBlock}.
  \item \textbf{\Block{DLWP}}---contains photon energy distributions for all photon production reactions. The \Block{DLWP} exists if \nxs{6}$\neq0$. See \Sectionref{sec:DLWBlock}.
  \item \textbf{\Block{YP}}---contains a list of MT identifiers of neutron reaction cross sections required as photon production yield multipliers. The \Block{YP} exists if \nxs{6}$\neq0$. See \Sectionref{sec:YPBlock}.
  \item \textbf{\Block{FIS}}---contains the total fission cross section tabulated on the ESZ energy grid. The \Block{FIS} exists if \jxs{21}$\neq0$. See \Sectionref{sec:FISBlock}.
  \item \textbf{\Block{UNR}}---contains the unresolved resonance range probability tables. The \Block{UNR} exists if \jxs{23}$\neq0$. See \Sectionref{sec:UNRBlock}.
\end{enumerate}

\subsection{\NXS\ Array}\label{sec:NXSContinuousEnergyNeutron}

\begin{ThreePartTable}
  \begin{TableNotes}
  \item[$\dagger$] \label{tn:Reserved} These entries are reserved for the use of transport codes (i.e., MCNP).
  \end{TableNotes}
  \begin{NXSTable}{continuous-energy neutron}
      1  & ---        & Length of second block of data (\XSS\ array) \\
      2  & \var{ZA}   & $1000*Z+A$ \\
      3  & \var{NES}  & Number of energies \\
      4  & \var{NTR}  & Number of reactions excluding elastic scattering \\
      5  & \var{NR}   & Number of reactions having secondary neutrons excluding elastic scattering \\
      6  & \var{NTRP} & Number of photon production reactions \\
         & \ldots     & \\
      8  & \var{NPCR} & Number of delayed neutron precurser families \\
         & \ldots     & \\
      14 &            & Reserved\tnotex{tn:Reserved} \\
      15 &            & Reserved\tnotex{tn:Reserved} \\
      16 &            & Reserved\tnotex{tn:Reserved} \\
    \label{tab:NXSContinuousEnergyNeutron}
  \end{NXSTable}
\end{ThreePartTable}

\subsection{\JXS\ Array}\label{sec:JXSContinuousEnergyNeutron}
\begin{JXSTable}{continuous-energy neutron}
    1  & \var{ESZ}   & Energy table \\
    2  & \var{NU}    & Fission $\nu$ data \\
    3  & \var{MTR}   & \texttt{MT} array \\
    4  & \var{LQR}   & $Q$-value array \\
    5  & \var{TYR}   & Reaction type array \\
    6  & \var{LSIG}  & Table of cross section locators \\
    7  & \var{SIG}   & Cross sections \\
    8  & \var{LAND}  & Table of angular distribution locators \\
    9  & \var{AND}   & Angular distributions \\
    10 & \var{LDLW}  & Table of energy distribution locators \\
    11 & \var{DLW}   & Energy distributions \\
    12 & \var{GPD}   & Photon production data \\
    13 & \var{MTRP}  & Photon production \texttt{MT} array \\
    14 & \var{LSIGP} & Table of photon production cross section locators \\
    15 & \var{SIGP}  & Photon production cross sections \\
    16 & \var{LANDP} & Table of photon production angular distribution locators \\
    17 & \var{ANDP}  & Photon production angular distributions \\
    18 & \var{LDLWP} & Table of photon production energy distribution locators \\
    19 & \var{DLWP}  & Photon production energy distributions \\
    20 & \var{YP}    & Table of yield multipliers \\
    21 & \var{FIS}   & Total fission cross section \\
    22 & \var{END}   & Last word of this table \\
    23 & \var{LUNR}  & Probability tables \\
    24 & \var{DNU}   & Delayed $\overline{\nu}$ data \\
    25 & \var{BDD}   & Basic delayed data ($\lambda$'s, probabilities) \\
    26 & \var{DNEDL} & Table of energy distribution locators \\
    27 & \var{DNED}  & Energy distributions \\
       & \ldots      & \\
    32 & ---         & 
  \label{tab:JXSContinuousEnergyNeutron}
\end{JXSTable}

\subsection{Format of Individual Data Blocks}
\subsubsection{\textsf{ESZ} Block}\label{sec:ESZBlock}
The format of the \Block{ESZ} is given in \Tableref{tab:ESZBlock}.
\begin{ThreePartTable}
  \begin{TableNotes}
  \item[$\dagger$] \label{tn:DisappearanceXS} The disappearance cross section is defined in \cite[Appendix B]{Trkov:2011ENDF--0} as \MT{101}
  \end{TableNotes}
  \begin{BlockTable}{ESZ}
    \startblock{ESZ}            & $E(l), l=1,\ldots, N_{E}$           & Energies \\
    \startblock{ESZ} + $N_{E}$  & $\sigma_{t}(l), l=1,\ldots, N_{E}$  & Total cross section \\
    \startblock{ESZ} + $2N_{E}$ & $\sigma_{a}(l), l=1,\ldots, N_{E}$  & Total neutron disappearance cross section\tnotex{tn:DisappearanceXS} \\
    \startblock{ESZ} + $3N_{E}$ & $\sigma_{el}(l), l=1,\ldots, N_{E}$ & Elastic cross section \\
    \startblock{ESZ} + $4N_{E}$ & $H_{ave}(l), l=1,\ldots, N_{E}$      & Average Heating numbers
    \label{tab:ESZBlock}
  \end{BlockTable}
  \begin{tablenotes}
    \note \startblock{ESZ} is index of the \XSS\ array where the \Block{ESZ} starts, \jxs{1},  and $N_{E}$ is the number of energy energy points, \nxs{3}.
  \end{tablenotes}
\end{ThreePartTable}

\subsubsection{\textsf{NU} Block}\label{sec:NUBlock}
There are four possibilities for the \Block{NU}:
\begin{enumerate}
  \item {\bfseries\sffamily No \Block{NU}.} \\
    This happens when \jxs{2}=0.
  \item {\bfseries\sffamily Either prompt or total \nubar\ is given (but not both).} \\
    The \aceArray{NU} array begins at location \xss{KNU} where \var{KNU}=\jxs{2}.
  \item {\bfseries\sffamily Both prompt and total \nubar\ are given.} \\
    The prompt \aceArray{NU} array begins at \xss{KNU} where \var{KNU}=\jxs{2}; the total \aceArray{NU} array begins at \xss{KNU} where {\sffamily \var{KNU} = \jxs{2} + ABS(\xss{\jxs{2}})+1}
  \item {\bfseries\sffamily Delayed \nubar\ is given.} \\
    The delayed \nubar\ array begins at \xss{KNU} where \var{KNU}=\jxs{24}. Delayed \nubar\ must be given in tabulated as described below in \Tableref{tab:NUBlockTabulated}.
\end{enumerate}

The format of the \Block{NU} has two forms (if it exists); polynomial (see \Tableref{tab:NUBlockPolynomial}) and tabulated (see \Tableref{tab:NUBlockTabulated}). The format is specified by the \var{LNU} flag located in the \XSS\ array at index \var{KNU} where \var{KNU} is defined above.
\begin{BlockTable}[---Polynomial function form]{NU}
  \var{KNU}   & \var{LNU}=1                     & Polynomial function flag \\
  \var{KNU}+1 & $N_{C}$                   & Number of coefficients \\
  \var{KNU}+2 & $C(l), l=1,\ldots, N_{C}$ & Coefficients
  \label{tab:NUBlockPolynomial}
\end{BlockTable}
When using the polynomial function form of the \aceArray{NU} array, \nubar\ is reconstructed as
\begin{equation}
  \nubar(E) = \sum_{l=1}^{N_{C}} C(l)E^{l-1},
  \label{eq:nubarPolynomialReconstruction}
\end{equation}
where the energy, $E$, is given in \si{\MeV}.

\begin{ThreePartTable}
  \begin{TableNotes}
  \item[$\dagger$] \label{tn:scheme} If $N_{R}=0$, \var{NBT} and \var{INT} are omitted and linear-linear interpolation is assumed.
  \end{TableNotes}
  \begin{BlockTable}[---Tabulated form]{NU}
    \var{KNU}                   & \var{LNU}=2                  & Tabulated data flag \\
    \var{KNU}+1                 & $N_{R}$                      & Number of interpolation regions \\
    \var{KNU}+2                 & \var{NBT}$(l), l=1,\ldots,N_{R}$   & ENDF interpolation parameters \\
    \var{KNU}+2+$N_{R}$         & \var{INT}$(l), l=1,\ldots,N_{R}$   & ENDF interpolation scheme\tnotex{tn:scheme} \\
    \var{KNU}+2+$2N_{R}$       & $N_{E}$                      & Number of energies \\
    \var{KNU}+3+$2N_{R}$       & $E(l),l=1,\ldots,N_{E}$      & Tabulated energy points \\
    \var{KNU}+3+$2N_{R}+N_{E}$ & $\nubar(l),l=1,\ldots,N_{E}$ & Tabulated \nubar\ values
    \label{tab:NUBlockTabulated}
  \end{BlockTable}
\end{ThreePartTable}

If delayed neutron data exist (when \jxs{24}>0), the precursor distribution format is given as in \Tableref{tab:DelayedPrecursorDistribution}. The decay constant for the first group \var{DEC}$_{1}$ is given at \xss{\jxs{25}}. The precursor distribution immediately follows as described in \Tableref{tab:DelayedPrecursorDistribution}. The indices (locators) of the \XSS\ array where each precursor distribution begins (\startblock{DNU}) can found using the format described in \Sectionref{sec:LDLWBlock} and \Sectionref{sec:DLWBlock}, where \var{LED}=\jxs{26} and \var{NMT}=\nxs{8}.
\begin{ThreePartTable}
  \begin{TableNotes}
    \item[$\dagger$] \label{tn:schemeDelayedPrecursors} If $N_{R}=0$, \var{NBT} and \var{INT} are omitted and linear-linear interpolation is assumed.
  \end{TableNotes}
  \begin{XSSTable}{Delayed \nubar\ precursor distribution.}
    \startblock{DNU}                   & \var{DEC}$_{i}$                  & Decay constant for the $i$-th group \\
    \startblock{DNU}+1                 & $N_{R}$                          & Number of interpolation regions \\
    \startblock{KNU}+2                 & \var{NBT}$(l), l=1,\ldots,N_{R}$ & ENDF interpolation parameters\tnotex{tn:schemeDelayedPrecursors} \\
    \startblock{KNU}+2+$N_{R}$         & \var{INT}$(l), l=1,\ldots,N_{R}$ & ENDF interpolation scheme \\
    \startblock{DNU}+2+$2N_{R}$       & $N_{E}$                          & Number of energies \\
    \startblock{DNU}+3+$2N_{R}$       & $E(l),l=1,\ldots,N_{E}$          & Tabulated energy points \\
    \startblock{DNU}+3+$2N_{R}+N_{E}$ & $P(l),l=1,\ldots,N_{E}$          & Corresponding probabilities
    \label{tab:DelayedPrecursorDistribution}
  \end{XSSTable}
  \begin{tablenotes}
    \note \startblock{DNU} is the index of the \XSS\ array where the delayed \nubar\ precursor distribution begins; the first one is at \startblock{DNU}=\jxs{25}.
  \end{tablenotes}
\end{ThreePartTable}

\subsubsection{\textsf{MTR} \& \textsf{MTRP} Blocks}\label{sec:MTRBlock}\label{sec:MTRPBlock}
The format of the \Block{MTR} (for incident neutron reactions) and \Block{MTRP} (for photon production reactions) is given in \Tableref{tab:MTRBlock}. The starting index depends on whether it is the \Block{MTR} or \Block{MTRP} and are given in \Tableref{tab:LMT_NMT}.
\begin{table}[h!] \centering
  \begin{tabular}[h]{lll}
    \toprule
    Block & \var{LMT} & \var{NMT} \\
    \midrule
    \var{MTR} & \jxs{3} & \nxs{4} \\
    \var{MTRP} & \jxs{13} & \nxs{6} \\
    \bottomrule
  \end{tabular}
  \caption{\var{LMT} and \var{NMT} values for the \Block{MTR} and \Block{MTR}.}
  \label{tab:LMT_NMT}
\end{table}

\begin{BlockTable}{MTR \textnormal{\&} MTRP}
  \var{LMT} & \MT$_{1}$ & First ENDF Reaction available \\
  \var{LMT}+1 & \MT$_{2}$ & Second ENDF Reaction available \\
  \ldots \\
  \var{LMT}+\var{NMT}+1 & \MT$_{\var{NMT}}$ & Last ENDF reaction available
  \label{tab:MTRBlock}
\end{BlockTable}

For the \Block{MTR}, \MT$_{1},\ldots,\MT_{\var{NMT}}$ are standard ENDF \MT numbers; that is, \MT=16=$(n,2n)$; \MT=17=$(n,3n)$; etc. For a complete listing of \MT\ numbers, see \cite[Appendix B]{Trkov:2011ENDF--0}.

For the \Block{MTRP}, the \MT\ numbers are somewhat arbitrary. To understand the scheme used for numbering the photon production \MT s, it is necessary to realize that in the ENDF format, more than one photon can be produced by a particular neutron reaction that is itself specified by a single \MT. Each of these photons is produced with an individual energy-dependent cross section. For example, \MT 102 (radiatiive capture) might be responsible for 40 photons, each with its own cross section, angular distribution, and energy distribution. We need 40 photon \MT s to represent the data; the \MT s are numbered \textsf{1002001}, \textsf{1002002}, \ldots, \textsf{1002040}. Therefore, if ENDF \MT\ $N$ is responsible for $M$ photons, we shall number the photon \MT s \textsf{1000*$N$+1}, \textsf{1000*$N$+2}, \ldots, \textsf{1000*$N$+$M$}.

\subsubsection{\textsf{LQR} Block}\label{sec:LQRBlock}
The format of the \Block{LQR}, containing the reaction-specific $Q$-values, is given in \Tableref{tab:LQRBlock}. The index at the start of the \Block{LQR}, \startblock{LQR}=\jxs{4}. The number of reactions, \var{NMT}, is the same through the \ACE\ Table, \var{NMT}=\nxs{4}.
\begin{BlockTable}{LQR}
  \startblock{LQR}           & $Q_{1}$         & $Q$-value for reaction \MT$_{1}$ \\
  \startblock{LQR}+1         & $Q_{2}$         & $Q$-value for reaction \MT$_{2}$ \\
  \ldots \\
  \startblock{LQR}+\var{NMT}-1 & $Q_{\var{NMT}}$ & $Q$-value for reaction \MT$_{\var{NMT}}$
  \label{tab:LQRBlock}
\end{BlockTable}

\subsubsection{\textsf{TYR} Block}\label{sec:TYRBlock}
The format of the \Block{TYR} is given in \Tableref{tab:TYRBlock}. The index at the start of the \Block{TYR}, \startblock{TYR}=\jxs{5}. The number of reactions, \var{NMT}, is the same through the \ACE\ Table, \var{NMT}=\nxs{4}.

\begin{BlockTable}{TYR}
  \startblock{TYR}             & \var{TY}$_{1}$         & Neutron release for reaction \MT$_{1}$ \\
  \startblock{TYR}+1           & \var{TY}$_{2}$         & Neutron release for reaction \MT$_{2}$ \\
  \ldots \\
  \startblock{TYR}+\var{NMT}-1 & \var{TY}$_{\var{NMT}}$ & Neutron release for reaction \MT$_{\var{NMT}}$
  \label{tab:TYRBlock}
\end{BlockTable}
The possible values of \var{TY} are $\pm 1$, $\pm 2$, $\pm 3$, $\pm 4$, $\pm 19$, 0, and integers greater than 100 in absolute value; the sign indicates the system for scattering: negative=\CM, positive=\LAB. Thus if \var{TY}$_{i}$=+3, three neutrons are released for reaction \MT$_{i}$ and the data on the cross section tables used to determine the exiting neutrons' angles are given in the \LAB\ frame of reference. \var{TY}=19 indicates fission. The number of secondary neutrons released is determined from the fission \nubar\ data found in the \Block{NU}. \var{TY}$_{i}$=0 indicates absorption (ENDF reactions \MT>100); no neutrons are released. $\|\var{TY}_{i}\|>100$ signifies reactions other than fission that have energy-dependent neutron multiplicities. The number of secondary neutrons released is determined from the yield data found in the \Block{DLW}. The \MT$_{i}$s are given in the \Block{MTR}.

\subsubsection{\textsf{LSIG}\& \textsf{LSIGP} Blocks}\label{sec:LSIGBlock}\label{sec:LSIGPBlock}
The \Block{LSIG} and \Block{LSIGP} give the locators for cross section array for each reaction \MT. A locator is a \emph{relative} index in the \XSS\ array where some piece of data. In this case, the data is the cross section values. The format of the \Block{LSIG} (for incident neutron cross sections) and \Block{LSIGP} (for photon production cross sections) is given in \Tableref{tab:LSIGBlock}. The format for the incident neutron cross section arrays is given in \Sectionref{sec:SIGBlock}. The format for the photon production cross sections is given in \Sectionref{sec:SIGPBlock}.

All locators are relative to \jxs{7} for the \Block{LSIG} or \jxs{15} for the \Block{LSIGP}. That is, \var{LXS}=\jxs{7} for the \Block{LSIG} and \var{LXS}=\jxs{15} for the \Block{LSIGP}. So the actual cross section data begins at the index \LOC{A}+\var{LXS}. The \MT s are given in the \Block{MTR} and the \Block{MTRP} for the \Block{LSIG} and the \Block{LSIGP} respectively. \LOC{A}{i} must be monotonically increasing.
\begin{BlockTable}{LSIG \textnormal{\&} LSIGP}
  \var{LXS}             & \LOC{A}{1}=1       & Location of cross sections for reaction \MT$_{1}$ \\
  \var{LXS}+1           & \LOC{A}{2}         & Location of cross sections for reaction \MT$_{2}$ \\
  \ldots \\
  \var{LXS}+\var{NMT}-1 & \LOC{A}{\var{NMT}} & Location of cross sections for reaction \MT$_{\var{NMT}}$
  \label{tab:LSIGBlock}
\end{BlockTable}

\subsubsection{\textsf{SIG} Block}\label{sec:SIGBlock}
The \Block{SIG} contains the incident neutron cross section data. (The photon production cross section is in the \Block{SIGP}.) The format of the \Block{SIG} is given in \Tableref{tab:SIGBlock}. The cross section data begins at the index specified by the locator from the \Block{LSIG}; the format for which is given in \Tableref{tab:CrossSectionArray}.
\begin{ThreePartTable}
  \begin{LOCTable}{\Block{SIG}}
    \var{LXS}+\LOC{A}{1}-1         & Cross section array for reaction \MT$_{1}$ \\
    \var{LXS}+\LOC{A}{2}-1         & Cross section array for reaction \MT$_{2}$ \\
    \ldots \\
    \var{LXS}+\LOC{A}{\var{NMT}}-1 & Cross section array for reaction \MT$_{\var{NMT}}$
    \label{tab:SIGBlock}
  \end{LOCTable}
  \begin{tablenotes}
    \note The number of cross section arrays \var{NMT}=\nxs{4}.
  \end{tablenotes}
\end{ThreePartTable}

The \LOC{A}{i} values are given in the \Block{LSIG} and are all relative to \jxs{7}. The energy grid index $\var{IE}_{i}$ corresponds to the first energy in the grid at which a cross section is given. The \MT$_{i}$s are defined in the \Block{MTR}.
\begin{ThreePartTable}
\begin{XSSTable}{Cross section array for the $i$-th reaction.}
  \var{LXS} + \LOC{A}{i}-1                  & $\var{IE}_{i}$                                 & Energy grid index for reaction \MT$_{i}$ \\
  \var{LXS} + \LOC{A}{i}                    & $N_{E,i}$                                      & Number of consecutive entries for \MT$_{i}$ \\
  \multirow{2}{*}{\var{LXS} + \LOC{A}{i}+1} & $\sigma_{i}[E(l)]$ for                         & \multirow{2}{0.65\linewidth}{Cross section for reaction \MT$_{i}$} \\
                                            & $l=\var{IE}_{i},\ldots,\var{IE}_{i}+N_{E,i}-1$ & 
  \label{tab:CrossSectionArray}
\end{XSSTable}
\begin{tablenotes}
   \note The energy grid, $E(l)$ is given in the \Block{ESZ}.
\end{tablenotes}
\end{ThreePartTable}

\subsubsection{\textsf{LAND} Block}\label{sec:LANDBlock}
The \Block{LAND} contains locators for the angular distributions for all reactions producing secondary neutrons. The \Block{LAND} always exists and begins at \startblock{LAND}=\jxs{8}. All locators (\LOC{B}) are relative \jxs{9}; that is, the angular distribution begins at \jxs{9}+\LOC{B}$_{i}$. The \LOC{B}{i} locators must be monotonically increasing. The format of the \Block{LAND} is given in \Tableref{tab:LANDBlock}.
\begin{ThreePartTable}
\begin{BlockTable}{LAND}
  \startblock{LAND} & \LOC{B}{1}=1 & Location of angular distribution data for elastic scattering reaction \\
  \startblock{LAND}+1 & \LOC{B}{2} & Location of angular distribution data for reaction \MT$_{1}$ \\
  \ldots \\
  \startblock{LAND}+\var{NMT} & \LOC{B}{\var{NMT}} & Location of angular distribution data for reaction \MT$_{\var{NMT}}$
  \label{tab:LANDBlock}
\end{BlockTable}
\begin{tablenotes}
  \note \startblock{LAND}=\jxs{8} and \var{NMT}=\nxs{5} is the number of reactions (excluding elastic scattering).
\end{tablenotes}
\end{ThreePartTable}

\subsubsection{\textsf{AND} Block}\label{sec:ANDBlock}
The \Block{AND} contains angular distribution data for all reactions that produce secondary neutrons. The format of the \Block{AND} is given in \Tableref{tab:ANDBlock}. The angular distribution data begins at the index specified by the locator \LOC{B} from the \Block{LAND}. If \LOC{B}{i}=0 (given in the \Block{LAND}), no angular distribution data are given for reaction $i$ and isotropic scattering is assumed in either the \LAB\ or \CM\ system. The choice of \LAB\ or \CM\ system depends upon the value for reaction $i$ in the \Block{TYR}. If \LOC{B}{i}=-1 no angular distribution data are given for reaction $i$ in the \Block{AND}. The angular distribution data are specified through \law{44} in the \Block{DLW}. 

\begin{ThreePartTable}
\begin{LOCTable}{\Block{AND}}
  \jxs{9}+\LOC{B}{1}-1         & Angular distribution array for elastic scattering \\
  \jxs{9}+\LOC{B}{2}-1         & Angular distribution array for reaction \MT$_{1}$ \\
  \jxs{9}+\LOC{B}{\var{NMT}}-1 & Angular distribution array for reaction \MT$_{\var{NMT}}$
  \label{tab:ANDBlock}
\end{LOCTable}
\begin{tablenotes}
  \note The format for the angular distribution of the $i$-th array is given in \Tableref{tab:AngularDistributionArray}.
\end{tablenotes}
\end{ThreePartTable}

\begin{XSSTable}{Angular distribution array for the $i$-th reaction}
  \jxs{9}+\LOC{B}{i}-1     & $N_{E}$                     & Number of energies at which angular distributions are tabulated.  \\
  \jxs{9}+\LOC{B}{i}       & $E(l),l=1,\ldots,N_{E}$     & Energy grid \\
  \jxs{9}+\LOC{B}{i}$+N_{E}$ & $L_{C}(l),l=1,\ldots,N_{E}$ & Location of tables associated with $E(l)$
  \label{tab:AngularDistributionArray}
\end{XSSTable}

The angular distribution arrays (\Tableref{tab:AngularDistributionArray}) contains additional locators, $L_{C}$; the sign of these locators is a flag:
\begin{itemize}
  \item if $L_{C}(l)>0$, then $L_{C}(l)$ points to a 32 equiprobable bin distribution (see \Tableref{tab:32EquiprobableBinDistribution});
  \item if $L_{C}(l)<0$, then $L_{C}(l)$ points to a tabulated angular distribution (see \Tableref{tab:TabulatedAngularDistribution});
  \item if $L_{C}(l)=0$, then distribution is isotropic and no further data is needed.
\end{itemize}

\begin{XSSTable}{Format for the 32 equiprobable bin distribution}
  \multirow{2}{*}{\jxs{9}+$|L_{C}(l)|-1$} & $P(1,K)$        & 32 equiprobable cosine bins for scattering  \\
                                          & $K=1,\ldots,33$ & at energy $E(l)$.
  \label{tab:32EquiprobableBinDistribution}
\end{XSSTable}

\begin{ThreePartTable}
  \begin{TableNotes}
    \item[$\dagger$] \label{tn:ANDInterpolationFlag}
      \begin{description}[font=\ttfamily]
        \item[0] histogram interpolation,
        \item[1] linear-linear interpolation
      \end{description}
  \end{TableNotes}
\begin{XSSTable}{Format for the tabulated angular distribution.}
  \var{LDAT}$_{l}+1$ & \var{JJ}                                & Interpolation flag\tnotex{tn:ANDInterpolationFlag} \\
  \var{LDAT}$_{l}+2$ & $N_{P}$                                 & Number of points in the distribution \\
  \var{LDAT}$_{l}+3$ & $CS_{\mathrm{out}}(j),j=1,\ldots,N_{P}$ & Cosine scattering angular grid \\
  \var{LDAT}$_{l}+4$ & $\PDF(j),j=1,\ldots,N_{P}$              & Probability density function \\
  \var{LDAT}$_{l}+5$ & $\CDF(j),j=1,\ldots,N_{P}$              & Cumulative density function
  \label{tab:TabulatedAngularDistribution}
\end{XSSTable}
\begin{tablenotes}
  \note \var{LDAT}$_{l}=\jxs{9}+|L_{C}(l)|-1$
\end{tablenotes}
\end{ThreePartTable}

\subsubsection{\textsf{LDLW} \& \textsf{LDLWP} Blocks}\label{sec:LDLWBlock}\label{sec:LDLWPBlock}
The \Block{LDLW} and \Block{LDLW} give the locators for the energy distribution for every reaction that produces secondary neutrons or secondary photons (respectively). The format of the \Block{LDLW} (for secondary neutrons) and \Block{LDLW} (for secondary photons) is given in \Tableref{tab:LDLWBlock}. The locators for the delayed neutron precursors (see \Sectionref{sec:NUBlock}) also use the same format. The format for the distribution arrays is given in \Sectionref{sec:DLWBlock}.

The \Block{LDLW} exists if \nxs{5}$\neq0$ while the \Block{LDLWP} exists if \nxs{6}$\neq0$. The starting index, \var{LED}, depends on what data is being read; the starting values and the number of locators, \var{NMT}, are given in \Tableref{tab:LEDNMT}.

\begin{table}[h!] \centering
  \begin{tabular}{lll}
    \toprule
    Block            & \var{LED} & \var{NMT} \\
    \midrule
    \var{LDLW}       & \jxs{10}  & \nxs{5} \\
    \var{LDLWP}      & \jxs{18}  & \nxs{6} \\
    delayed neutrons & \jxs{26}  & \nxs{8} \\
    \bottomrule
  \end{tabular}
  \caption{LED and NMT values for the \Block{LDLW} and \Block{LDLWP}.}
  \label{tab:LEDNMT}
\end{table}

\begin{ThreePartTable}
\begin{BlockTable}{LDLW}
  \var{LED}             & \LOC{C}{1}         & Location of energy distribution data for reaction \MT$_{1}$ or group 1 (if delayed neutron) \\
  \var{LED}+1           & \LOC{C}{2}         & Location of energy distribution data for reaction \MT$_{2}$ or group 2 (if delayed neutron) \\
  \ldots \\
  \var{LED}+\var{NMT}-1 & \LOC{C}{\var{NMT}} & Location of energy distribution data for reaction \MT$_{\var{NMT}}$ or group \var{NMT} (if delayed neutron)
  \label{tab:LDLWBlock}
\end{BlockTable}
\begin{tablenotes}
  \note The \LOC{C}{i} must be monotonically increasing.
\end{tablenotes}
\end{ThreePartTable}

All locators point to data \emph{relative} to \var{JED} (see \Sectionref{sec:DLWBlock}) in the \XSS\ array. The \MT\ values are given in the \Block{MTR} for \Block{LDLW} or \Block{MTRP} for \Block{LDLWP}.

\subsubsection{\textsf{DLW} \& \textsf{DLWP} Blocks}\label{sec:DLWBlock}\label{sec:DLWPBlock}
The \Block{DLW} contains secondary energy distributions for all reactions producing secondary neutrons---expect for elastic scattering. The \Block{DLWP} contains secondary energy distribution for all photon-producing reactions. Both the \Block{DLW} and \Block{DLWP} have the same format. The energy distributions are given starting with a locator, \LOC{C}, which were given in the \Block{LDLW} and \Block{LDLWP}. The locators are relative to the \var{JED} parameter. The value for \var{JED} and \var{NMT} (the number of reactions) is dependent on whether it is the \Block{DLW} or \Block{DLWP}. These values are given in \Tableref{tab:JED_NMT}.
\begin{table}[h!] \centering
  \begin{tabular}{lll}
    \toprule
    Block            & \var{JED} & \var{NMT} \\
    \midrule
    \var{DLW}        & \jxs{11}  & \nxs{5} \\
    \var{DLWP}       & \jxs{19}  & \nxs{6} \\
    delayed neutrons & \jxs{27}  & N/A \\
    \bottomrule
  \end{tabular}
  \caption{\var{JED} and \var{NMT} for the \Block{DLW} and \Block{DLW}.}
  \label{tab:JED_NMT}
\end{table}

\begin{LOCTable}{\Block{DLW}}
  \var{JED}+\LOC{C}{1}-1 & Energy distribution array for reaction \MT$_{1}$ \\
  \var{JED}+\LOC{C}{2}-1 & Energy distribution array for reaction \MT$_{2}$ \\
  \ldots \\
  \var{JED}+\LOC{C}{\var{NMT}}-1 & Energy distribution array for reaction \MT$_{\var{NMT}}$
  \label{tab:DLWBlock}
\end{LOCTable}

The $i$-th array has the form shown in 
\begin{ThreePartTable}
  \begin{TableNotes}
  \item[$\dagger$] \label{tn:LNW} If \var{LNW}$_{i}=0$ then \var{LAW}$_{1}$ is used regardless of other circumstantes.
  \item[$\ddagger$] \label{tn:EnergyDistributionInterpolationScheme} If $N_{R}=0$, \var{NBT} and \var{INT} are omitted and linear-linear interpolation is assumed.
  \item[$\ast$] \label{tn:EnergyDistributionProbability} If the particle energy $E<E(1)$, then $P(E)=P(1)$. If $E>E(N_{E})$, then $P(E)=P(N_{E})$. If more than one law is given, then \var{LAW}$_{1}$ is used only if $\xi<P(E)$ where $\xi$ is a random number between 0 and 1.
  \end{TableNotes}
  \begin{XSSTable}{Format for the secondary energy distribution.}
    \var{JED}+\LOC{C}{i}-1                & \var{LNW}$_{1}$                  & Location of next law.\tnotex{tn:LNW} \\
    \var{JED}+\LOC{C}{i}                  & \var{LAW}$_{1}$                  & Name of this law \\
    \var{JED}+\LOC{C}{i}+1                & \var{IDAT}$_{1}$                 & Location of data for this law relative to \var{JED} \\
    \var{JED}+\LOC{C}{i}+2                & $N_{R}$                          & Number of interpolation regions to define law applicability regime \\
    \var{JED}+\LOC{C}{i}+3                & \var{NBT}$(l), l=1,\ldots,N_{R}$ & ENDF interpolation parameters \\
    \var{JED}+\LOC{C}{i}+3+$N_{R}$        & \var{INT}$(l), l=1,\ldots,N_{R}$ & ENDF interpolation scheme\tnotex{tn:EnergyDistributionInterpolationScheme} \\
    \var{JED}+\LOC{C}{i}+3+$2N_{R}$       & $N_{E}$                          & Number of energies \\
    \var{JED}+\LOC{C}{i}+4+$2N_{R}$       & $E(l),l=1,\ldots,N_{E}$          & Tabulated energy points \\
    \var{JED}+\LOC{C}{i}+4+$2N_{R}+N_{E}$ & $P(l),l=1,\ldots,N_{E}$          & Probability of law validity\tnotex{tn:EnergyDistributionProbability} \\
    \var{JED}+\var{IDAT}$_{1}-1$          & \var{LDAT}$(l),l,\ldots,L$       & Law data for \var{LAW}$_{1}$. \\
    \var{JED}+\var{LNW}$_{1}-1$           & \var{LNW}$_{2}$                  & Location of next law \\
    \var{JED}+\var{LNW}$_{1}$             & \var{LAW}$_{2}$                  & Name of this law \\
    \var{JED}+\var{LNW}+1                 & \var{IDAT}$_{2}$                 & Location of data for this law relative to \var{JED} \\
    \ldots
    \label{tab:EnergyDistributionArray}
  \end{XSSTable}
\end{ThreePartTable}

The format for the law data depends on the law. The length, $L$, of the law data array, \var{LDAT}, is determined from parameters with \var{LDAT}. The various \var{LDAT} arrays and their formats are given in the following tables. Laws 2 (\Tableref{tab:LAW2}) and 4 (\Tableref{tab:LAW4}) are used to describe spectra of secondary photons from neutron collisions. All laws---except for Law 2---are usd to describe the spectra of scattered neutrons. 

In the following tables, we provide relative locations of data in the \var{LDAT} array rather than the absolute locations in the \XSS\ array. \Tableref{tab:EnergyDistributionArray} defines the starting location of the \var{LDAT} array within the \XSS\ array.

\subsubsubsection{\var{LAW}=1---Tabular Equiprobable Energy Bins}\label{sec:LAW1}
\begin{ThreePartTable}
  \begin{TableNotes}
    \item[$\dagger$] \label{tn:LAW1InterpolationScheme} If $N_{R}=0$, \var{NBT} and \var{INT} are omitted and linear-linear interpolation is assumed.
    \item[$\ddagger$] \label{tn:EoutTables} $E_{\mathrm{out}}$ tables consist of \var{NET} boundaries of \var{NET}-1 equally likely energy intervals. Linear-linear interpolation is used between intervals.
  \end{TableNotes}
  \begin{LAWTable}{\var{LAW}=1 (From ENDF Law 1)}
    \var{LDAT}(1)                 & $N_{R}$                                            & Number of interpolation regions between tables of $E_{\mathrm{out}}$ \\
    \var{LDAT}(2)                 & \var{NBT}$(l), l=1,\ldots,N_{R}$                   & ENDF interpolation parameters \\
    \var{LDAT}(2+$N_{R}$)         & \var{INT}$(l), l=1,\ldots,N_{R}$                   & ENDF interpolation scheme\tnotex{tn:LAW1InterpolationScheme} \\
    \var{LDAT}(2+$2N_{R}$)        & $N_{E}$                                            & Number of incident energies tabulated \\
    \var{LDAT}(3+$2N_{R}$)        & $E_{\mathrm{in}}(l),l=1,\ldots,N_{E}$               & List of incident energies for which $E_{\mathrm{out}}$ is tabulated \\
    \var{LDAT}(3+$2N_{R}+N_{E}$)  & \var{NET}                                          & Number of outgoing energies in each $E_{\mathrm{out}}$ table \\
    \var{LDAT}(4+$2*N_{R}+N_{E}$) & $E_{\mathrm{out}_{1}}(l),l=1,\ldots,\var{NET}$     &  $E_{\mathrm{out}}$ tables\tnotex{tn:EoutTables} \\
                                  & $E_{\mathrm{out}_{2}}(l),l=1,\ldots,\var{NET}$     & \\
                                  & \ldots \\
                                  & $E_{\mathrm{out}_{N_{E}}}(l),l=1,\ldots,\var{NET}$ &
    \label{tab:LAW1}
  \end{LAWTable}
\end{ThreePartTable}

\subsubsubsection{\var{LAW}=2---Discrete Photon Energy}\label{sec:LAW2}
\begin{ThreePartTable}
\begin{LAWTable}{\var{LAW}=2---Discrete Photon Energy}
  \var{LDAT}(1) & \var{LP} & Indicator of whether the photon is a primary or non-primary photon \\
  \var{LDAT}(2)& \var{EG} & Photon energy or binding energy
  \label{tab:LAW2}
\end{LAWTable}
\begin{tablenotes}
  \note If \var{LP}=0 or \var{LP}=1, the photon energy is \var{EG}. If \var{LP}=2, the photon energy is 
    \[\var{EG}+\left(\frac{\var{AWR}}{\var{AWR}+1}\right)E_{N}\]
    where \var{AWR} is the atomic weight ratio and $E_{N}$ is the incident neutron energy.
\end{tablenotes}
\end{ThreePartTable}

\subsubsubsection{\var{LAW}=3---Level Scattering}\label{sec:LAW3}
\begin{align}
  \var{LDAT}(1) &= \left( \frac{A+1}{A} \right)|Q| \\
  \var{LDAT}(2) &= \left( \frac{A}{A+1} \right)^{2} \\
  E_{\mathrm{out}}^{\mathrm{CM}} &= \var{LDAT}(2)*(E-\var{LDAT}(1))
  \label{eq:LAW3}
\end{align}
where
\begin{align*}
  E_{\mathrm{out}}^{\mathrm{CM}} &= \textnormal{outgoing \CM\ energy} \\
  E &= \textnormal{incident energy} \\
  A &= \textnormal{atomic weight ratio} \\
  Q &= Q\textnormal{-value}
\end{align*}

The outgoing neutron energy in the laboratory system is:
\begin{equation}
  E_{\mathrm{out}}^{\mathrm{LAB}} = E_{\mathrm{out}}^{\mathrm{CM}} +\left\{ E+2\mu_{\mathrm{CM}}(A+1)(E\cdot E_{\mathrm{out}}^{\mathrm{CM}})^{1/2} \right\}/(A+1)^{2}
  \label{eq:Law3EoutLAB}
\end{equation}
where $\mu_{\mathrm{CM}}$ is the cosine of the \CM\ scattering angle

\subsubsubsection{\var{LAW}=4---Continuous Tabular Distribution}\label{sec:LAW4}
\begin{ThreePartTable}
  \begin{TableNotes}
    \item[$\dagger$] \label{tn:LAW4InterpolationScheme} If $N_{R}=0$, \var{NBT} and \var{INT} are omitted and linear-linear interpolation is assumed.
    \item[$\ddagger$] \label{tn:LAW4Locators} Relative to \jxs{11} (neutron reactions), \jxs{19} (photon-producing reactions), or \jxs{27} (delayed neutrons).
  \end{TableNotes}
  \begin{LAWTable}{\var{LAW}=4 (From ENDF-6 \var{LAW}=1)}
    \var{LDAT}(1)                & $N_{R}$                          & Interpolation scheme between tables of $E_{\mathrm{out}}$ \\
    \var{LDAT}(2)                & \var{NBT}$(l), l=1,\ldots,N_{R}$ & ENDF interpolation parameters \\
    \var{LDAT}(2+$N_{R}$)        & \var{INT}$(l), l=1,\ldots,N_{R}$ & ENDF interpolation scheme\tnotex{tn:LAW4InterpolationScheme} \\
    \var{LDAT}(2+$2N_{R}$)       & $N_{E}$                          & Number of energies at which distributions are tabulated \\
    \var{LDAT}(3+$2N_{R}$)       & $E(l),l=1,\ldots,N_{E}$           & Incident neutron energies \\
    \var{LDAT}(3+$2N_{R}+N_{E}$) & $\var{L}(l),l=1,\ldots,N_{E}$           & Locations of distributions\tnotex{tn:LAW4Locators}
    \label{tab:LAW4}
  \end{LAWTable}
\end{ThreePartTable}

The data associated with each incident neutron energy begins at the location $\var{L}(l)$. The format for the data is given in \Tableref{tab:LAW4Distribution}, where for $E(1)$ let \var{K}=3+$2N_{R}+2N_{E}$. 
\begin{LAWTable}{Secondary energy distribution for each incident energy in \var{LAW}=4.}
  \multicolumn{3}{c}{\bfseries Data for $\mathbf{E(1)}$} \\
  \var{LDAT}(\var{K}) & $\var{INTT}'$ & Interpolation parameter \\
  \var{LDAT}(\var{K}+1) & $N_{p}$ & Number of points in the distribution \\
  \var{LDAT}(\var{K}+2) & $E_{\mathrm{out}}(l),l=1,\ldots,N_{p}$ & outgoing energy grid \\
  \var{LDAT}(\var{K}+$2+N_{p}$) & $\mathrm{PDF}(l),l=1,\ldots,N_{p}$ & Probability Density Function \\
  \var{LDAT}(\var{K}+$2+2N_{p}$) & $\mathrm{CDF}(l),l=1,\ldots,N_{p}$ & Cumulative Density Function \\
  \midrule
  \multicolumn{3}{c}{{\bfseries Data for $\mathbf{E(2)}$}---same format for $E(1)$} \\
  \multicolumn{3}{c}{\ldots} \\
  \multicolumn{3}{c}{{\bfseries Data for $\mathbf{E(N_{E})}$}---same format for $E(1)$}
  \label{tab:LAW4Distribution}
\end{LAWTable}

The first element in the data is $\var{INTT}'$ or the interpolation parameter, which is a combination of two other parameters:
\begin{enumerate}
  \item the number of discrete photon lines, $N_{D}$, and
  \item the interpolation scheme for the subsequent data, $\var{INTT}$.
\end{enumerate}
\var{INTT} has two valid values:
\begin{description}
  \item[\var{INTT}=1] histogram distribution, and 
  \item[\var{INTT}=2] linear-linear distribution.
\end{description}
If the value of $\var{INTT}'>10$, then
\begin{equation*}
  \var{INTT}' = 10N_{D}+\var{INTT}
\end{equation*}
where \var{INTT} is the interpolation scheme and the first $N_{D}$ values of $N_{p}$ points describe discrete photon lines. The remaining ($N_{p}-N_{D}$) values describe a continuous distribution. In this way, the distribution may be discrete, continuous, or a discrete distribution superimposed upon a continuous background.

\subsubsubsection{\var{LAW}=5---General Evaporation Spectrum}\label{sec:LAW5}
\begin{LAWTable}{\var{LAW}=5 (From ENDF-6, \MF=5, \var{LF}=5)}
  \var{LDAT}(1)                 & $N_{R}$                          & \multirow{2}{0.65\linewidth}{Interpolation scheme between $T$'s} \\
  \var{LDAT}(2)                 & \var{NBT}$(l), l=1,\ldots,N_{R}$ & \\
  \var{LDAT}(2+$N_{R}$)         & \var{INT}$(l), l=1,\ldots,N_{R}$ & \\
  \cmidrule{1-3}
  \var{LDAT}(2+$2N_{R}$)        & $N_{E}$                          & Number of incident energies tabulated \\
  \var{LDAT}(3+$2N_{R}$)        & $E(l),l=1,\ldots,N_{E}$          & Incident energy table \\
  \var{LDAT}(3+$2N_{R}+N_{E}$)  & $\theta(l),l=1,\ldots,N_{E}$     & Effective temperature tabulated on incident energies \\
  \var{LDAT}(3+$2N_{R}+2N_{E}$) & \var{NET}                        & Number of $X$'s tabulated \\
  \var{LDAT}(4+$2N_{R}+2N_{E}$) & $X(l),l=1,\ldots,\var{NET}$      & Equiprobable bins
  \label{tab:LAW5}
\end{LAWTable}
\begin{equation}
  E_{\mathrm{out}} = X(\xi)\theta(E)
  \label{eq:LAW5}
\end{equation}
where:
\begin{description}
  \item[$X(\xi)$]  is a randomly sampled table of $X$'s; 
  \item[$\theta(E)$] is the effective temperature tabulated on incident energy; and 
  \item[$E$] is the incident energy.
\end{description}

\subsubsubsection{\var{LAW}=7---Simple Maxwellian Fission Spectrum}\label{sec:LAW7}
\begin{LAWTable}{\var{LAW}=7 (From ENDF-6, \MF=5, \var{LF}=7)}
  \var{LDAT}(1)                 & $N_{R}$                          & \multirow{3}{0.65\linewidth}{Interpolation scheme between $T$'s} \\
    \var{LDAT}(2)                 & \var{NBT}$(l), l=1,\ldots,N_{R}$ & \\
    \var{LDAT}(2+$N_{R}$)         & \var{INT}$(l), l=1,\ldots,N_{R}$ & \\
    \cmidrule{1-3}
    \var{LDAT}(2+$2N_{R}$)        & $N_{E}$                          & Number of incident energies tabulated \\
    \var{LDAT}(3+$2N_{R}$)        & $E(l),l=1,\ldots,N_{E}$          & Incident energy table \\
    \var{LDAT}(3+$2N_{R}+N_{E}$)  & $\theta(l),l=1,\ldots,N_{E}$     & Effective temperature tabulated on incident energies \\
    \var{LDAT}(3+$2N_{R}+2N_{E}$) & $U$                              & Restriction energy
  \label{tab:LAW7}
\end{LAWTable}

The outgoing energy, $E_{\mathrm{out}}$, can be calculated as
\begin{equation}
  f(E\rightarrow E_{\mathrm{out}}) = \frac{\sqrt{E_{\mathrm{out}}}}{I}\ e^{-E_{\mathrm{out}}/\theta(E)}
  \label{eq:LAW7f}
\end{equation}
where:
\begin{description}
  \item[$I$] is the normalization constant
    \begin{equation}
      I = \theta^{{3/2}} \frac{\sqrt{\pi}}{2} \erf\left( \sqrt{(E-U)/\theta} \right) - \sqrt{(E-U)/\theta}\ e^{-(E-U)/\theta},
      \label{eq:LAW7I}
    \end{equation}
  \item[$\theta$] is tabulated as a function of incident energy, $E$; and
  \item[$U$] is a constant introduced to define the proper upper limit for the final particle energy such that $0\leq E_{\mathrm{out}} \leq (E-U)$.
\end{description}

\subsubsubsection{\var{LAW}=9---Evaporation Spectrum}\label{sec:LAW9}
\begin{LAWTable}{\var{LAW}=9 (From ENDF-6, \MF=5, \var{LF}=9)}
  \var{LDAT}(1)                 & $N_{R}$                          & \multirow{3}{0.65\linewidth}{Interpolation scheme between $T$'s} \\
    \var{LDAT}(2)                 & \var{NBT}$(l), l=1,\ldots,N_{R}$ & \\
    \var{LDAT}(2+$N_{R}$)         & \var{INT}$(l), l=1,\ldots,N_{R}$ & \\
    \cmidrule{1-3}
    \var{LDAT}(2+$2N_{R}$)        & $N_{E}$                          & Number of incident energies tabulated \\
    \var{LDAT}(3+$2N_{R}$)        & $E(l),l=1,\ldots,N_{E}$          & Incident energy table \\
    \var{LDAT}(3+$2N_{R}+N_{E}$)  & $\theta(l),l=1,\ldots,N_{E}$     & Effective temperature tabulated on incident energies \\
    \var{LDAT}(3+$2N_{R}+2N_{E}$) & $U$                              & Restriction energy
  \label{tab:LAW9}
\end{LAWTable}
The outgoing energy, $E_{\mathrm{out}}$, can be calculated as
\begin{equation}
  f(E\rightarrow E_{\mathrm{out}}) = \frac{\sqrt{E_{\mathrm{out}}}}{I}\ e^{-E_{\mathrm{out}}/\theta(E)}
  \label{eq:LAW9f}
\end{equation}
where:
\begin{description}
  \item[$I$] is the normalization constant
    \begin{equation}
      I = \theta^{2}\left[ 1-e^{-(E-U)/\theta}\left( 1+\frac{E-U}{\theta} \right) \right],
      \label{eq:LAW9I}
    \end{equation}
  \item[$\theta$] is tabulated as a function of incident energy, $E$; and
  \item[$U$] is a constant introduced to define the proper upper limit for the final particle energy such that $0\leq E_{\mathrm{out}} \leq (E-U)$.
\end{description}
\textbf{Note:} \Equationref{eq:LAW9f} is the same as \Equationref{eq:LAW7f}; just the definitions of $I$ in \Equationref{eq:LAW7I} and \Equationref{eq:LAW9I} are different.

\subsubsubsection{\var{LAW}=11---Energy Dependent Watt Spectrum}\label{sec:LAW11}
\begin{LAWTable}{\var{LAW}=11 (From ENDF-6, \MF=5, \var{LF}=11)}
  \var{LDAT}(1)                        & $N_{R_{a}}$                              & \multirow{3}{0.65\linewidth}{Interpolation scheme between $a$'s} \\
  \var{LDAT}(2)                        & \var{NBT}$_{a}(l), l=1,\ldots,N_{R_{a}}$ & \\
  \var{LDAT}(2+$N_{R_{a}}$)            & \var{INT}$_{a}(l), l=1,\ldots,N_{R_{a}}$ & \\
  \var{LDAT}(2+$2N_{R_{a}}$)           & $N_{E_{a}}$                              & Number of incident energies tabulated for $a(E_{\mathrm{in}})$ table \\
  \var{LDAT}(3+$2N_{R_{a}}$)           & $E_{a}(l),l=1,\ldots,N_{E_{a}}$          & Incident energy table \\
  \var{LDAT}(3+$2N_{R_{a}}+N_{E_{a}}$) & $a(l),l=1,\ldots,N_{E_{a}}$              & Tabulated $a$'s \\
  \cmidrule{1-3}
  \multicolumn{3}{l}{let $\var{L}=3+2\left( N_{R_{a}}+N_{E_{a}} \right)$} \\
  \var{LDAT}(\var{L})                          & $N_{R_{b}}$                              & \multirow{3}{0.65\linewidth}{Interpolation scheme between $b$'s} \\
  \var{LDAT}(\var{L}+1)                        & \var{NBT}$_{b}(l), l=1,\ldots,N_{R_{b}}$ & \\
  \var{LDAT}(\var{L}+1+$N_{R_{b}}$)            & \var{INT}$_{b}(l), l=1,\ldots,N_{R_{b}}$ & \\
  \var{LDAT}(\var{L}+1+$2N_{R_{b}}$)           & $N_{E_{b}}$                              & Number of incident energies tabulated for $b(E_{\mathrm{in}})$ table \\
  \var{LDAT}(\var{L}+2+$2N_{R_{b}}$)           & $E_{b}(l),l=1,\ldots,N_{E_{b}}$          & Incident energy table \\
  \var{LDAT}(\var{L}+2+$2N_{R_{b}}+N_{E_{b}}$) & $b(l),l=1,\ldots,N_{E_{b}}$              & Tabulated $b$'s \\
  \var{LDAT}(\var{L}+2+$2N_{R_{b}}+2N_{E_{b}}$ & $U$                                      & Rejection energy
  \label{tab:LAW11}
\end{LAWTable}
The outgoing energy, $E_{\mathrm{out}}$, can be calculated as
\begin{equation}
  f(E\rightarrow E_{\mathrm{out}}) = \frac{e^{-E_{\mathrm{out}}/a}}{I} \sinh\left( \sqrt{bE_{\mathrm{out}}} \right)
  \label{eq:LAW11f}
\end{equation}
where:
\begin{description}[font=\textnormal]
  \item[$I$] is the normalization constant
    \begin{multline}
      I = \frac{1}{2}\sqrt{\frac{\pi a^{3}b}{4}}e^{(ab/4)} \left[ \erf\left( \sqrt{\frac{E-U}{a}} - \sqrt{\frac{ab}{4}} \right) + \erf\left( \sqrt{\frac{E-U}{a}} + \sqrt{\frac{ab}{4}} \right) \right] \\
      - a e^{-(E-U)/a} \sinh \sqrt{b(E-U)};
      \label{eq:LAW11I}
    \end{multline}
  \item[$a$ and $b$] are tabulated energy-dependent parameters; and
  \item[$U$] is a constant introduced to define the proper upper limit for the final particle energy such that $0\leq E_{\mathrm{out}} \leq (E-U)$.
\end{description}

\subsubsubsection{\var{LAW}=22---Tabular Linear Functions of Incident Energy Out}\label{sec:LAW22}
\begin{LAWTable}{\var{LAW}=22 (From UK Law 2)}
  \var{LDAT}(1)                 & $N_{R}$                          & \multirow{3}{0.65\linewidth}{Interpolation parameters} \\
    \var{LDAT}(2)                 & \var{NBT}$(l), l=1,\ldots,N_{R}$ & \\
    \var{LDAT}(2+$N_{R}$)         & \var{INT}$(l), l=1,\ldots,N_{R}$ & \\
    \cmidrule{1-3}
    \var{LDAT}(2+$2N_{R}$)        & $N_{E}$                          & Number of incident energies tabulated \\
    \var{LDAT}(3+$2N_{R})$ & $E_{\mathrm{in}}(l),l=1,\ldots,N_{E}$ & Tabulated incident energies for $E_{\mathrm{out}}$ tables \\
    \var{LDAT}(3+$2N_{R}+N_{E})$ & $\LOC{E}(l),l=1,\ldots,N_{E}$ & Locators of $E_{\mathrm{out}}$ tables\\
    \cmidrule{1-3}
    \multicolumn{3}{l}{Data for $E_{\mathrm{in}}(1)$ Let $\var{L}=3+2N_{R}+2N_{E}$:} \\
    \var{LDAT}(\var{L})                   & $\var{NF}_{1}$ \\
    \var{LDAT}(\var{L}+1)                 & $P_{1k}, k=1,\ldots,\var{NF}_{1}$ \\
    \var{LDAT}(\var{L}+1+\var{NF}$_{1}$)  & $T_{1k}, k=1,\ldots,\var{NF}_{1}$ \\
    \var{LDAT}(\var{L}+1+2\var{NF}$_{1}$) & $C_{1k}, k=1,\ldots,\var{NF}_{1}$ \\
    \cmidrule{1-3}
    \multicolumn{3}{l}{Data for $E_{\mathrm{in}}(2)$:} \\
    \ldots
  \label{tab:LAW22}
\end{LAWTable}
Tables of $P_{ik}, C_{ik}$, and $T_{ik}$ are given at a number of incident energies, $E_{\mathrm{in}}$. If
\begin{equation}
  E_{\mathrm{in}}(l) \leq E < E_{\mathrm{in}}(l+1)
\end{equation}
then the secondary neutron energy is:
\begin{equation}
  E_{\mathrm{out}} = C_{ik}\left( E-T_{ik} \right),
  \label{eq:LAW22Eout}
\end{equation}
where $k$ is chosen according to
\begin{equation}
  \sum_{j=1}^{k} P_{ij} < \xi \leq \sum_{k=1}^{k+1}P_{ij}
  \label{eq:LAW22Sum}
\end{equation}
for a given random number, $\xi\in[0,1)$.

\subsubsubsection{\var{LAW}=24}\label{sec:LAW24}
\begin{LAWTable}{\var{LAW}=24 (From UK Law 6)}
  \var{LDAT}(1)          & $N_{R}$                               & \multirow{3}{0.65\linewidth}{Interpolation scheme between $T$'s} \\
    \var{LDAT}(2)          & \var{NBT}$(l), l=1,\ldots,N_{R}$      & \\
    \var{LDAT}(2+$N_{R}$)  & \var{INT}$(l), l=1,\ldots,N_{R}$      & \\
    \cmidrule{1-3}
    \var{LDAT}(2+$2N_{R}$) & $N_{E}$                               & Number of incident energies tabulated \\
    \var{LDAT}(3+$2N_{R}$) & $E_{\mathrm{in}}(l),l=1,\ldots,N_{E}$ & List of incident energies for which $T$ is tabulated \\
    \var{LDAT}(3+$2N_{R}+N_{E}$) & \var{NET} & Number of outgoing values in each table \\
    \var{LDAT}(4+$2N_{R}+N_{E}$) & $T_{1}(l),l=1,\ldots,\var{NET}$ & \multirow{4}{0.65\linewidth}{Tables have \var{NET} boundaries with \var{NET}-1 equally likely intervals. Linear-linear interpolation is used between intervals.} \\
                                 & $T_{2}(l),l=1,\ldots,\var{NET}$ & \\
                                 & \ldots & \\
                                 & $T_{N_{E}}(l),l=1,\ldots,\var{NET}$ & \\
  \label{tab:LAW24}
\end{LAWTable}
The outgoing energy, $E_{\mathrm{out}}$ can be calculated as:
\begin{equation}
  E_{\mathrm{out}} = T_{k}(l)*E
  \label{eq:LAW24}
\end{equation}
where:
\begin{description}
  \item[$T_{k}(l)$]  is sampled from the tables and
  \item[$E$] is the incident energy.
\end{description}

\subsubsubsection{\var{LAW}=44---Kalbach-87 Formalism}\label{sec:LAW44}
\begin{ThreePartTable}
  \begin{TableNotes}
    \item[$\dagger$] \label{tn:LAW44InterpolationScheme} If $N_{R}=0$, \var{NBT} and \var{INT} are omitted and linear-linear interpolation is assumed.
    \item[$\ddagger$] \label{tn:LAW44Locators} Relative to \jxs{11} (neutron reactions), \jxs{19} (photon-producing reactions), or \jxs{27} (delayed neutrons).
  \end{TableNotes}
  \begin{LAWTable}{\var{LAW}=44 (From ENDF-6 \MF=6 \var{LAW}=1, \var{LANG}=2)}
    \var{LDAT}(1)                & $N_{R}$                          & Interpolation scheme between tables of $E_{\mathrm{out}}$ \\
    \var{LDAT}(2)                & \var{NBT}$(l), l=1,\ldots,N_{R}$ & ENDF interpolation parameters \\
    \var{LDAT}(2+$N_{R}$)        & \var{INT}$(l), l=1,\ldots,N_{R}$ & ENDF interpolation scheme\tnotex{tn:LAW44InterpolationScheme} \\
    \var{LDAT}(2+$2N_{R}$)       & $N_{E}$                          & Number of energies at which distributions are tabulated \\
    \var{LDAT}(3+$2N_{R}$)       & $E(l),l=1,\ldots,N_{E}$          & Incident neutron energies \\
    \var{LDAT}(3+$2N_{R}+N_{E}$) & $\var{L}(l),l=1,\ldots,N_{E}$    & Locations of distributions\tnotex{tn:LAW44Locators}
    \label{tab:LAW44}
  \end{LAWTable}
\end{ThreePartTable}

The data associated with each incident neutron energy begins at the location $\var{L}(l)$. The format for the data is given in \Tableref{tab:LAW44Distribution}, where for $E(1)$ let \var{K}=3+$2N_{R}+2N_{E}$. 
\begin{LAWTable}{Secondary energy distribution for each incident energy in \var{LAW}=44}
  \multicolumn{3}{c}{\bfseries Data for $\mathbf{E(1)}$} \\
  \var{LDAT}(\var{K})            & $\var{INTT}'$                          & Interpolation parameter \\
  \var{LDAT}(\var{K}+1)          & $N_{p}$                                & Number of points in the distribution \\
  \var{LDAT}(\var{K}+2)          & $E_{\mathrm{out}}(l),l=1,\ldots,N_{p}$ & outgoing energy grid \\
  \var{LDAT}(\var{K}+$2+N_{p}$)  & $\mathrm{PDF}(l),l=1,\ldots,N_{p}$     & Probability Density Function \\
  \var{LDAT}(\var{K}+$2+2N_{p}$) & $\mathrm{CDF}(l),l=1,\ldots,N_{p}$     & Cumulative Density Function \\
  \var{LDAT}(\var{K}+$2+3N_{p}$) & $R(l),l=1,\ldots,N_{p}$                & Precompound fraction $r$ \\
  \var{LDAT}(\var{K}+$2+4N_{p}$) & $A(l),l=1,\ldots,N_{p}$                & Angular distribution slope value $a$ \\
  \cmidrule{1-3}
  \multicolumn{3}{c}{{\bfseries Data for $\mathbf{E(2)}$}---same format for $E(1)$} \\
  \multicolumn{3}{c}{\ldots} \\
  \multicolumn{3}{c}{{\bfseries Data for $\mathbf{E(N_{E})}$}---same format for $E(1)$}
  \label{tab:LAW44Distribution}
\end{LAWTable}

The first element in the data is $\var{INTT}'$ or the interpolation parameter, which is a combination of two other parameters:
\begin{enumerate}
  \item the number of discrete photon lines, $N_{D}$, and
  \item the interpolation scheme for the subsequent data, $\var{INTT}$.
\end{enumerate}
\var{INTT} has two valid values:
\begin{description}
  \item[\var{INTT}=1] histogram distribution, and 
  \item[\var{INTT}=2] linear-linear distribution.
\end{description}
If the value of $\var{INTT}'>10$, then
\begin{equation*}
  \var{INTT}' = 10N_{D}+\var{INTT}
\end{equation*}
where \var{INTT} is the interpolation scheme and the first $N_{D}$ values of $N_{p}$ points describe discrete photon lines. The remaining ($N_{p}-N_{D}$) values describe a continuous distribution. In this way, the distribution may be discrete, continuous, or a discrete distribution superimposed upon a continuous background.

The angular distributions for neutrons are then sampled from:
\begin{equation}
  p(\mu,E_{\mathrm{in}},E_{\mathrm{out}}) = \frac{1}{2}\frac{a}{\sinh(a)}\left[ \cosh(a\mu)+r\sinh(a\mu) \right].
  \label{eq:LAW44p}
\end{equation}

\subsubsubsection{\var{LAW}=61---Like \var{LAW}=44, but tabular angular distribution instead of Kalbach-87}\label{sec:LAW61}
\begin{ThreePartTable}
  \begin{TableNotes}
    \item[$\dagger$] \label{tn:LAW61InterpolationScheme} If $N_{R}=0$, \var{NBT} and \var{INT} are omitted and linear-linear interpolation is assumed.
    \item[$\ddagger$] \label{tn:LAW61Locators} Relative to \jxs{11} (neutron reactions), \jxs{19} (photon-producing reactions), or \jxs{27} (delayed neutrons).
  \end{TableNotes}
  \begin{LAWTable}{\var{LAW}=61}
    \var{LDAT}(1)                & $N_{R}$                          & Number of interpolation regions \\
    \var{LDAT}(2)                & \var{NBT}$(l), l=1,\ldots,N_{R}$ & ENDF interpolation parameters \\
    \var{LDAT}(2+$N_{R}$)        & \var{INT}$(l), l=1,\ldots,N_{R}$ & ENDF interpolation scheme\tnotex{tn:LAW61InterpolationScheme} \\
    \var{LDAT}(2+$2N_{R}$)       & $N_{E}$                          & Number of energies at which distributions are tabulated \\
    \var{LDAT}(3+$2N_{R}$)       & $E(l),l=1,\ldots,N_{E}$          & Incident neutron energies \\
    \var{LDAT}(3+$2N_{R}+N_{E}$) & $\var{L}(l),l=1,\ldots,N_{E}$    & Locations of distributions\tnotex{tn:LAW61Locators}
    \label{tab:LAW61}
  \end{LAWTable}
\end{ThreePartTable}

The data associated with each incident neutron energy begins at the location $\var{L}(l)$. The format for the data is given in \Tableref{tab:LAW61Distribution}, where for $E(1)$ let \var{K}=3+$2N_{R}+2N_{E}$. 
\begin{LAWTable}{Secondary energy distribution for each incident energy in \var{LAW}=61}
  \multicolumn{3}{c}{\bfseries Data for $\mathbf{E(1)}$} \\
  \var{LDAT}(\var{K})            & $\var{INTT}'$                          & Interpolation parameter \\
  \var{LDAT}(\var{K}+1)          & $N_{p}$                                & Number of points in the distribution \\
  \var{LDAT}(\var{K}+2)          & $E_{\mathrm{out}}(l),l=1,\ldots,N_{p}$ & outgoing energy grid \\
  \var{LDAT}(\var{K}+$2+N_{p}$)  & $\mathrm{PDF}(l),l=1,\ldots,N_{p}$     & Probability Density Function \\
  \var{LDAT}(\var{K}+$2+2N_{p}$) & $\mathrm{CDF}(l),l=1,\ldots,N_{p}$     & Cumulative Density Function \\
  \var{LDAT}(\var{K}+$2+3N_{p}$) & $\var{LC}(l),l=1,\ldots,N_{p}$         & Location of tables associated with incident energies $E(l)$. See \Tableref{tab:LAW61AngularDistribution}\\
  \cmidrule{1-3}
  \multicolumn{3}{c}{{\bfseries Data for $\mathbf{E(2)}$}---same format for $E(1)$} \\
  \multicolumn{3}{c}{\ldots} \\
  \multicolumn{3}{c}{{\bfseries Data for $\mathbf{E(N_{E})}$}---same format for $E(1)$}
  \label{tab:LAW61Distribution}
\end{LAWTable}
If the value of $\var{INTT}'>10$, then
\begin{equation*}
  \var{INTT}' = 10N_{D}+\var{INTT}
\end{equation*}
where \var{INTT} is the interpolation scheme and the first $N_{D}$ values of $N_{p}$ points describe discrete photon lines. The remaining ($N_{p}-N_{D}$) values describe a continuous distribution. In this way, the distribution may be discrete, continuous, or a discrete distribution superimposed upon a continuous background.

The $J$-th array for the tabular angular distribution has the form shown in \Tableref{tab:LAW61AngularDistribution}. For the angular distribution, the locators \var{L} are relative to \jxs{11} for neutron reactions or \jxs{19} for photon-producing reactions. Thus, 
\begin{align*}
  \var{L} &= \jxs{11} + |\var{LC}(J)|-1\ \text{(for neutron reactions)}, \\
  \var{L} &= \jxs{19} + |\var{LC}(J)|-1\ \text{(for photon-producing reactions)}. \\
\end{align*}
\begin{LAWTable}{Angular distribution for \var{LAW}=61}
  \var{LDAT}(\var{L}+1)          & \var{JJ}                                & Interpolation flag \\
  \var{LDAT}(\var{L}+2)          & $N_{P}$                                 & Number of points in the distribution \\
  \var{LDAT}(\var{L}+3)          & $CS_{\mathrm{out}}(j),j=1,\ldots,N_{P}$ & Cosine scattering angular grid \\
  \var{LDAT}(\var{L}+3+$N_{P}$)  & $\PDF(j),j=1,\ldots,N_{P}$              & Probability density function \\
  \var{LDAT}(\var{L}+3+$2N_{P}$) & $\CDF(j),j=1,\ldots,N_{P}$              & Cumulative density function
  \label{tab:LAW61AngularDistribution}
\end{LAWTable}

\subsubsubsection{\var{LAW}=66---$N$-body phase space distribution}\label{sec:LAW66}
\begin{LAWTable}{\var{LAW}=66 (From ENDF-6 \MF=6 \var{LAW}=6)}
  \var{LDAT}(1) & \var{NPSX} & Number of bodies in the phase space \\
  \var{LDAT}(2) & $A_{P}$ & Total mass ratio for the \var{NPSX} particles.
  \label{tab:LAW66}
\end{LAWTable}

The outgoing energy is
\begin{align}
  E_{\mathrm{out}} &= T(\xi)E_{i}^{\mathrm{max}} \\
  \intertext{where}
  E_{i}^{\mathrm{max}} &= \frac{A_{p}-1}{A_{p}}\left( \frac{A}{A+1}E_{\mathrm{in}}+Q \right) \\
  \intertext{and $T(\xi)$ is sampled from:}
  P_{i}(\mu,E_{\mathrm{in}},T) &= C_{n}\sqrt{T}\left( E_{i}^{\mathrm{max}}-T \right)^{3n/2-4}
  \label{eq:LAW66}
\end{align}

\subsubsubsection{\var{LAW}=67---Laboratory Angle-Energy Law}\label{sec:LAW67}
\begin{ThreePartTable}
  \begin{TableNotes}
    \item[$\dagger$] \label{tn:LAW67InterpolationScheme} If $N_{R}=0$, \var{NBT} and \var{INT} are omitted and linear-linear interpolation is assumed.
    \item[$\ddagger$] \label{tn:LAW67Locators} Relative to \jxs{11} (neutron reactions), \jxs{19} (photon-producing reactions), or \jxs{27} (delayed neutrons).
  \end{TableNotes}
  \begin{LAWTable}{\var{LAW}=67 (From ENDF-6 \MF=6 \var{LAW}=7)}
    \var{LDAT}(1)                & $N_{R}$                          & Interpolation scheme between tables of $E_{\mathrm{out}}$ \\
    \var{LDAT}(2)                & \var{NBT}$(l), l=1,\ldots,N_{R}$ & ENDF interpolation parameters \\
    \var{LDAT}(2+$N_{R}$)        & \var{INT}$(l), l=1,\ldots,N_{R}$ & ENDF interpolation scheme\tnotex{tn:LAW67InterpolationScheme} \\
    \var{LDAT}(2+$2N_{R}$)       & $N_{E}$                          & Number of energies at which distributions are tabulated \\
    \var{LDAT}(3+$2N_{R}$)       & $E(l),l=1,\ldots,N_{E}$          & Incident neutron energies \\
    \var{LDAT}(3+$2N_{R}+N_{E}$) & $\var{L}(l),l=1,\ldots,N_{E}$    & Locations of distributions\tnotex{tn:LAW67Locators}
    \label{tab:LAW67}
  \end{LAWTable}
\end{ThreePartTable}

The data associated with each distribution begins at location $\var{L}(l)$. The format for the data is given in \Tableref{tab:LAW67AngularDistribution}, where for $E(1)$ let $\var{K}=3+2N_{R}+2N_{e}$.
\begin{ThreePartTable}
  \begin{TableNotes}
    \item[$\dagger$] \label{tn:LAW67AngularInterpolationScheme} 
      \begin{description}
        \item[\var{INTMU}=1] histogram distribution,
        \item[\var{INTMU}=2] linear-linear distribution.
      \end{description}
  \end{TableNotes}
  \begin{LAWTable}{Angular distribution for \var{LAW}=67}
    \var{LDAT}(\var{K})   & \var{INTMU} & Interpolation scheme\tnotex{tn:LAW67AngularInterpolationScheme} \\
    \var{LDAT}(\var{K}+1) & \var{NMU}   & Number of secondary cosines \\
    \var{LDAT}(\var{K}+2) & $\var{XMU}(l),l=1,\ldots,\var{NMU}$ & Secondary cosines \\
    \var{LDAT}(\var{K}+2+\var{NMU}) & $\var{LMU}(l),l=1,\ldots,\var{NMU}$ & Locations of data for each secondary cosine. See \Tableref{tab:LAW67EnergyDistribution}
    \label{tab:LAW67AngularDistribution}
  \end{LAWTable}
\end{ThreePartTable}

The format for the secondary energy distribution (for each cosine bin, \var{XMU}) is given in \Tableref{tab:LAW67EnergyDistribution}. For the energy distribution, the locators, \var{LMU}, are relative to \jxs{11} or \jxs{19}. Thus,
\begin{align*}
  \var{L}_{l} &= \jxs{11}+\var{LMU}(l)\ \text{(for neutron reactions)}, \\
  \var{L}_{l} &= \jxs{19}+\var{LMU}(l)\ \text{(for photon-producing reactions)}.
\end{align*}
\begin{ThreePartTable}
  \begin{TableNotes}
    \item[$\dagger$] \label{tn:LAW67EnergyInterpolationScheme} 
      \begin{description}
        \item[\var{INTEP}=1] histogram distribution,
        \item[\var{INTEP}=2] linear-linear distribution.
      \end{description}
    \end{TableNotes}
  \begin{LAWTable}{Secondary energy distribution for each cosine bin in \var{LAW}=67}
    \var{LDAT}($\var{L}_{l}$) & \var{INTEP} & Interpolation parameter between secondary energies\tnotex{tn:LAW67EnergyInterpolationScheme} \\
    \var{LDAT}($\var{L}_{l}+1$) & \var{NPEP} & Number of secondary energies \\
    \var{LDAT}($\var{L}_{l}+2$) & $E_{P}(l),l=1,\ldots,\var{NPEP}$ & Secondary energy grid \\
    \var{LDAT}($\var{L}_{l}+2+\var{NPEP}$) & $\PDF(l),l=1,\ldots,\var{NPEP}$ & Probability density function \\
    \var{LDAT}($\var{L}_{l}+2+2\var{NPEP}$) & $\CDF(l),l=1,\ldots,\var{NPEP}$ & Cumulative density function
    \label{tab:LAW67EnergyDistribution}
  \end{LAWTable}
\end{ThreePartTable}

\subsubsubsection{Energy-Dependent Neutron Yields}
There are additional numbers to be found for neutrons in the \Block{DLW} and \Block{DLWP}. For those reactions with entries in the \Block{TYR} that are greater than \num{100} in absolute value, there must be neutron yields, $Y(E)$ provided as a function of neutron energy. The neutron yields are handled similarly to the average number of neutrons per fission, $\nu(E)$ that is given for the fission reactions. These yields are a part of the coupled energy-angle distributions given in File 6 of ENDF-6 data.

% The location in \XSS is:
% \begin{equation}
% \var{JED} + |\var{TY}_{i}|-101\ \text{Neutron yield data for reaction}\MT_{i}
%   \label{eq:NeutronYieldData}
% \end{equation}
% where
% \begin{align*}
%   \var{JED} &= \jxs{11} \\
%   \intertext{and}
%   i &\leq \text{number of reactions with negative angular distributions locators.}
% \end{align*}

The $i$-th array has the form given in \Tableref{tab:EnergyDependentNeutronYields}, where $\var{KY}=\var{JED}+|\var{TY}_{i}|-101$.
\begin{ThreePartTable}
  \begin{TableNotes}
    \item[$\dagger$] \label{tn:EDNYInterpolationScheme} If $N_{R}=0$, \var{NBT} and \var{INT} are omitted and linear-linear interpolation is assumed.
  \end{TableNotes}
  \begin{LAWTable}{Energy-Dependent Neutron Yields}
    \var{KY}                 & $N_{R}$                          & Number of interpolation regions \\
    \var{KY}+1               & \var{NBT}$(l), l=1,\ldots,N_{R}$ & ENDF interpolation parameters \\
    \var{KY}+1+$N_{R}$       & \var{INT}$(l), l=1,\ldots,N_{R}$ & ENDF interpolation scheme\tnotex{tn:EDNYInterpolationScheme} \\
    \var{KY}+1+2$N_{R}$      & $N_{E}$                          & Number of energies \\
    \var{KY}+2+2$N_{R}$      & $E(l),l=1,\ldots,N_{E}$          & Tabular energy points \\
    \var{KY}+2+$N_{R}+N_{E}$ & $Y(l),l=1,\ldots,N_{E}$          & Corresponding energy-dependent yields
    \label{tab:EnergyDependentNeutronYields}
  \end{LAWTable}
\end{ThreePartTable}

\subsubsection{\textsf{GPD} Block}\label{sec:GPDBlock}
The \Block{GPD} contains the \emph{total} photon production cross section, tabulated on the energy grid given in the \Block{ESZ}, the size of which is given by \nxs{3}. The \Block{GPD} only exists if \jxs{12}$\neq0$. 

There are \num{30} groups for the incident neutron energies, the boundaries of which are shown in \Tableref{tab:DiscreteNeutronEnergyBoundaries}. For each incident neutron energy group, the outgoing photon energies are discretized into \num{20} equiprobable energy groups, thus creating a $30\times20$ matrix. The outgoing energies are given in the \Block{GPD} as shown in \Tableref{tab:GPDBlock}. Note that this matrix is only used for older tables that do not provide expanded photon production data.
\begin{table}[h!] \centering
  \caption{Discrete neutron energy boundaries.}
  \begin{tabular}{rS[table-format = 1.3e1]|rS}
    \toprule
    \multirow{2}{*}{Group \#} & \multicolumn{1}{c}{Upper Boundary} & \multirow{2}{*}{Group \#} & \multicolumn{1}{c}{Upper Boundary} \\
                              & \multicolumn{1}{c}{(\si{\MeV})}    &                           & \multicolumn{1}{c}{(\si{\MeV})} \\
    \midrule
    1                         & 1.39E-10                     & 16                        & .184      \\
    2                         & 1.52E-7                      & 17                        & .303     \\
    3                         & 4.14E-7                      & 18                        & .500     \\
    4                         & 1.13E-6                      & 19                        & .823     \\
    5                         & 3.06E-6                      & 20                        & 1.353    \\
    6                         & 8.32E-6                      & 21                        & 1.738    \\
    7                         & 2.26E-5                      & 22                        & 2.232    \\
    8                         & 6.14E-5                      & 23                        & 2.865    \\
    9                         & 1.67E-4                      & 24                        & 3.68     \\
    10                        & 4.54E-4                      & 25                        & 6.07     \\
    11                        & 1.235E-3                     & 26                        & 7.79      \\
    12                        & 3.35E-3                      & 27                        & 10.      \\
    13                        & 9.23E-3                      & 28                        & 12.      \\
    14                        & 2.48E-2                      & 29                        & 13.5     \\
    15                        & 6.76E-2                      & 30                        & 15.      \\
    \bottomrule
  \end{tabular}
  \label{tab:DiscreteNeutronEnergyBoundaries}
\end{table}

The format of this Block is given in \Tableref{tab:GPDBlock}. The \XSS\ array index at the start of the \Block{GPD}, \startblock{GPD}=\jxs{12}.
\begin{BlockTable}{GPD}
  \startblock{GPD} & $\sigma_{\gamma}(l),l=1,\ldots,\var{NES}$  & Total photon production cross section \\
  \startblock{GPD}+\var{NES} & $E_{1}(K),K=1,20$ & \num{20} equiprobable outgoing photon energies for incident neutron $E<E_{N}(2)$ \\
  \startblock{GPD}+\var{NES}+20 & $E_{2}(K),K=1,20$ & \num{20} equiprobable outgoing photon energies for incident neutron $E_{N}(2) \leq E < E_{N}(3)$ \\
  \ldots \\
  \startblock{GPD}+\var{NES}+(i-1)*20 & $E_{i}(K),K=1,20$ & \num{20} equiprobable outgoing photon energies for incident neutron $E_{N}(i) \leq E < E_{N}(i+1)$ \\
  \ldots \\
  \startblock{GPD}+\var{NES}+(30-1)*20 & $E_{N}(K),K=1,20$ & \num{20} equiprobable outgoing photon energies for incident neutron $E \geq E_{N}(30)$
  \label{tab:GPDBlock}
\end{BlockTable}

\subsubsection{\textsf{SIGP} Block}\label{sec:SIGPBlock}
The \Block{SIGP} contains the photon production cross section data. The format of the \Block{SIGP} is given in \Tableref{tab:SIGPBlock}. The cross section data begins at the index specified by the locator, \LOC{A}{i}, given in the \Block{LSIG} (see \Sectionref{sec:LSIGBlock}). The \MT s are defined in the \Block{MTRP} (see \Sectionref{sec:MTRBlock}). All indices to the \XSS\ array are \emph{relative} to \jxs{15}. 
\begin{ThreePartTable}
\begin{BlockTable}{SIGP}
  \jxs{15}+\LOC{A}{1}-1         & \var{MFTYPE}$_{1}$ & Cross section array for reaction \MT$_{1}$ \\
  \jxs{15}+\LOC{A}{2}-1         & \var{MFTYPE}$_{2}$ & Cross section array for reaction \MT$_{2}$ \\
  \ldots \\
  \jxs{15}+\LOC{A}{\var{NMT}}-1 & \var{MFTYPE}$_{\var{NMT}}$ & Cross section array for reaction \MT$_{\var{NMT}}$
  \label{tab:SIGPBlock}
\end{BlockTable}
\begin{tablenotes}
  \note The number of photon production cross section arrays \var{NMT}=\nxs{6}.
\end{tablenotes}
\end{ThreePartTable}

The format of the $i$-th cross section array has two possible forms depending on the first number in the array, \var{MFTYPE}.
\begin{enumerate}
  \item If \var{MFTYPE}=12 or \var{MFTYPE}=16, yield data taken from ENDF File 12 or 6, respectively (see \Tableref{tab:PhotonProductionArray}). With this format, the photon production cross section can be constructed using \Equationref{eq:PhotonProductionConstruction};
  \begin{equation}
    \sigma_{\gamma,i}(E) = Y(E)*\sigma_{\var{MTMULT}}(E).
    \label{eq:PhotonProductionConstruction}
  \end{equation}
  \item If \var{MFTYPE}=13, cross section data from ENDF File 13 (see \Tableref{tab:PhotonProductionCrossSectionArray}).
\end{enumerate}

\begin{ThreePartTable}
  \begin{TableNotes}
  \item[$\dagger$] \label{tn:PPANBT} If $N_{R}=0$, \var{NBT} and \var{INT} are omitted and linear-linear interpolation is used.
  \end{TableNotes}
\begin{XSSTable}{Photon production array if \var{MFTYPE}=12 or 16}
  \jxs{15}+\LOC{A}{i}-1   & \var{MFTYPE}                                      & 12 or 16 \\
  \jxs{15}+\LOC{A}{i}     & \var{MTMULT}                                      & Neutron \MT\ whose cross section should multiply the yield \\
  \jxs{15}+\LOC{A}{i}+1   & $N_{R}$                                           & Number of interpolation regions \\
  \jxs{15}+\LOC{A}{i}+2   & \var{NBT}$(l), l=1,\ldots,N_{R}$                  & ENDF interpolation parameters\tnotex{tn:PPANBT} \\
  \jxs{15}+\LOC{A}{i}+2   & \multirow{2}{*}{\var{INT}$(l), l=1,\ldots,N_{R}$} & \multirow{2}{*}{ENDF interpolation scheme} \\
  \hfill +$N_{R}$         &                                                   & \\
  \jxs{15}+\LOC{A}{i}+2   & \multirow{2}{*}{$N_{E}$}                          & \multirow{2}{*}{Number of energies at which the yield is tabulated} \\
  \hfill +$2*N_{R}$       &                                                   & \\
  \jxs{15}+\LOC{A}{i}+3   & \multirow{2}{*}{$E(l),l=1,\ldots,N_{E}$}          & \multirow{2}{*}{Energies} \\
  \hfill +$2*N_{R}$       &                                                   & \\
  \jxs{15}+\LOC{A}{i}+3   & \multirow{2}{*}{$Y(l),l=1,\ldots,N_{E}$ }         & \multirow{2}{*}{Yields} \\
  \hfill +$2*N_{R}+N_{E}$ &                                                   & 
  \label{tab:PhotonProductionArray}
\end{XSSTable}
  
\end{ThreePartTable}

\begin{ThreePartTable}
\begin{XSSTable}{Photon production cross section array if \var{MFTYPE}=13}
  \jxs{15}+\LOC{A}{i}-1                  & \var{MFTYPE}                          & 13 \\
  \jxs{15}+\LOC{A}{i}                    & \var{IE}                              & Energy grid index \\
  \jxs{15}+\LOC{A}{i}+1                  & $N_{E}$                               & Number of consecutive entries \\
  \multirow{2}{*}{\jxs{15}+\LOC{A}{i}+2} & $\sigma_{\gamma,i}[E(K)],$            & \multirow{2}{*}{Photon production cross sections for reaction \MT$_{i}$} \\
                                         & $K=\var{IE},\ldots,\var{IE}+N_{E}-1$ & 
  \label{tab:PhotonProductionCrossSectionArray}
\end{XSSTable}
\begin{tablenotes}
  \note The \MT$_{i}$s are defined in the \Block{MTRP}.
\end{tablenotes}
\end{ThreePartTable}

\subsubsection{\textsf{LANDP} Block}\label{sec:LANDPBlock}

The \Block{LANDP} gives locator information for angular distribution arrays for photon production reactions and exists if $\nxs{6}\neq0$. All locators (\LOC{B}) in the \Block{LANDP} are \emph{relative} to \jxs{17}; that is, the angular distribution arrays begin at $\jxs{17}+\LOC{B}{i}$. The number of photon-producing reactions is \var{NMT}=\nxs{6}. The \LOC{B}{i} must be monotonically increasing. The \MT s are defined in the \Block{MTRP} (see \Sectionref{sec:MTRBlock}). The format of the \Block{LANDP} is given in \Tableref{tab:LANDPBlock}.

\begin{ThreePartTable}
\begin{BlockTable}{LANDP}
  \jxs{16} & \LOC{B}{1}=1 & Location of angular distribution data for reaction \MT$_{1}$ \\
  \jxs{16}+1 & \LOC{B}{2} & Location of angular distribution data for reaction \MT$_{2}$ \\
  \ldots \\
  \jxs{16}+\var{NMT}-1 & \LOC{B}{\var{NMT}} & Location of angular distribution data for reaction \MT$_{\var{NMT}}$ \\
  \label{tab:LANDPBlock}
\end{BlockTable}
\begin{tablenotes}
  \note The \LOC{B}{i} must be monotonically increasing. The format for the angular distribution of the $i$-th reaction is given in \Tableref{tab:SecondaryPhotonAngularDistribution}.
\end{tablenotes}
\end{ThreePartTable}

\subsubsection{\textsf{ANDP} Block}\label{sec:ANDPBlock}
The \Block{ANDP} contains angular distribution data for all photon-producing reactions and exists if $\nxs{6}\neq0$. The format of the \Block{ANDP} is given in \Tableref{tab:ANDPBlock}; the format of each angular distribution array is given in \Tableref{tab:SecondaryPhotonAngularDistribution}. The angular distribution data begins at the index specified by the locator, \LOC{B}, from the \Block{LANDP}; if $\LOC{B}{i}=0$, there are no angular distribution data given for reaction $i$ and isotropic scattering is assumed in the \LAB\ system. 
\begin{ThreePartTable}
\begin{LOCTable}{ANDP}
  \jxs{17}+\LOC{B}{1}-1         & Angular distribution array for reaction \MT$_{1}$ \\
  \jxs{17}+\LOC{B}{2}           & Angular distribution array for reaction \MT$_{2}$ \\
  \ldots \\
  \jxs{17}+\LOC{B}{\var{NMT}}-1 & Angular distribution array for reaction \MT$_{\var{NMT}}$
  \label{tab:ANDPBlock}
\end{LOCTable}
\begin{tablenotes}
  \note \var{NMT}=\nxs{6} is the number of photon-producing reactions.
\end{tablenotes}
\end{ThreePartTable}

\begin{ThreePartTable}
  \begin{TableNotes}
  \item[$\dagger$] \label{tn:CosineBinLocators} All values of $L_{C}(l)$ are \emph{relative} to \jxs{17}. If $L_{C}(l)=0$, no table is given for energy $E(l)$ and scattering is assumed to be isotropic in the \LAB\ system.
  \end{TableNotes}
\begin{XSSTable}{Angular distribution array for the $i$-th photon-producing reaction}
  \jxs{17}+\LOC{B}{i}-1       & $N_{E}$                      & Number of energies at which angular distributions are tabulated. \\
  \jxs{17}+\LOC{B}{i}         & $E(l),l=1,N_{E}$             & Energy grid \\
  \jxs{17}+\LOC{B}{i}+$N_{E}$ & $L_{C}(l),l=1,\ldots, N_{E}$ & Location of tables associated with $E(l)$\tnotex{tn:CosineBinLocators} \\
  \jxs{17}+$L_{C}(1)-1$       & $P_{1}(K),K=1,\ldots,33$     & 32 equiprobable cosine bins for scattering at energy $E(1)$ \\
  \jxs{17}+$L_{C}(2)-1$       & $P_{2}(K),K=1,\ldots,33$     & 32 equiprobable cosine bins for scattering at energy $E(2)$ \\
  \ldots \\
  \jxs{17}+$L_{C}(N_{E})-1$       & $P_{N_{E}}(K),K=1,\ldots,33$     & 32 equiprobable cosine bins for scattering at energy $E(N_{E})$
  \label{tab:SecondaryPhotonAngularDistribution}
\end{XSSTable}
\end{ThreePartTable}

\subsubsection{\textsf{YP} Block}\label{sec:YPBlock}
The \Block{YP} contains a list of \MT\ identifiers of neutron cross sections that are used as yield multipliers in \Equationref{eq:PhotonProductionConstruction} to calculate the photon production cross sections and are referenced by the \var{MTMULT} parameter in \Tableref{tab:PhotonProductionArray}. The \Block{YP} exists if $\nxs{6}\neq0$. The format of the \Block{YP} is given in \Tableref{tab:YPBlock}.
\begin{BlockTable}{YP}
  \jxs{20} & \var{NYP} & Number of neutron \MT s to follow \\
  \jxs{20}+1 & $\var{MTY}(l),l=1,\ldots,\var{NYP}$ & Neutron \MT s.
  \label{tab:YPBlock}
\end{BlockTable}

\subsubsection{\textsf{FIS} Block}\label{sec:FISBlock}
The \Block{FIS} contains the total fission cross section. The \Block{FIS} exists if $\jxs{21}\neq0$, but is generally not provided; the total fission cross section is redundant as the total fission cross section is the summation of first-, second-, third-, and fourth-chance fission (\MT=19, 20, 21, and 38);
\begin{equation}
  \sigma_{f,\mathrm{t}}(E) = \sigma_{(n,f)} + \sigma_{(n,nf)} + \sigma_{(n,2nf)} + \sigma_{(n,3nf)}.
  \label{eq:FissionSummation}
\end{equation}
The format of the \Block{FIS} is given in \Tableref{tab:FISBlock}.
\begin{ThreePartTable}
\begin{BlockTable}{FIS}
  \jxs{21} & \var{IE} & Energy grid index \\
  \jxs{21}+1 & $N_{E}$ & Number of consecutive entries \\
  \jxs{21}+2 & $\sigma_{f}[E(l)],K=\var{IE},\ldots,\var{IE}+N_{E}-1$ & Total fission cross sections
  \label{tab:FISBlock}
\end{BlockTable}
\begin{tablenotes}
  \note The energy $E(l)$ is given in the \Block{ESZ}.
\end{tablenotes}
\end{ThreePartTable}

\subsubsection{\textsf{UNR} Block}\label{sec:UNRBlock}
The \Block{UNR} contains the unresolved resonance range probability tables. It exists if $\jxs{23}\neq0$ and begins at location \jxs{23} in \texttt{XSS}. The \Block{UNR} has several flags that have special meaning:

\begin{description}[labelindent=1em]
  \item[\var{ILF}] The \var{ILF} flag is the inelastic competition flag.% If this flag is less than zero, the inelastic cross section is zero within the entire unresolved energy range. If this flag 
    \begin{description}
      \item[$\var{ILF}<0$] The inelastic cross section is zero within the entire unresolved energy range.
      \item[$\var{ILF}>0$] The value of \var{ILF} is a special \MT\ number whose tabulation is the sum of the inelastic levels. 
      \item[$\var{ILF}=0$] The sum of the contribution of the inelastic reactions will be made using a balance relationship involving the smooth cross sections.
    \end{description}
    An exception to this scheme is typically made when there is only one inelastic level within the unresolved energy range, because the flag can then just be set to its \MT\ number and the special tabulation is not needed.
  \item[\var{IOA}] The \var{IOA} is the other absorption flag for determining the contribution of ``other absorptions'' (no neutron out or destruction reactions).
    \begin{description}
      \item[$\var{IOA}<0$] The ``other absorption'' cross section is zero within the entire unresolved resonance range. 
      \item[$\var{IOA}>0$] The value of \var{IOA} is a special \MT\ number whose tabulation is the sum of the ``other absorption'' reactions. 
      \item[$\var{IOA}=0$] The sum of the contribution of the ``other absorption'' reactions will be made using a balanced relationship involving the smooth cross sections.
    \end{description}
    An exception to this scheme is typically made when there is only one ``other absorption'' reaction within the unresolved energy range, because the flag can then just be set to its \MT\ number and the special tabulation is not needed. 
  \item[\var{IFF}] The \var{IFF} is the factors flag.
    \begin{description}
      \item[$\var{IFF}=0$] The tabulations in the probability tables are cross sections.
      \item[$\var{IFF}=1$] The tabulations in the probability tables are factors that must be multiplied by the corresponding ``smooth'' cross sections to obtain the actual cross sections.
    \end{description}
  %\item[$P(i,j,k)$] $P(i,j,k)$ are the values that make up the probability table.
\end{description}

\par
The format of the \Block{UNR} is given in \Tableref{tab:UNRBlock}. The $P(i,j,k)$ values, where
\begin{itemize}
  \item $i=1,\ldots,N$,
  \item $j=1,\ldots,6$,
  \item $k=1,\ldots,M$,
\end{itemize}
are what make up the probability tables. The argument $j$ has special meaning depending on its value as shown in \Tableref{tab:Argumentj}.
\begin{table}[h!bt] \centering
  \caption{Possible values for the $j$ argument.}
  \begin{tabular}{ll}
    \toprule
    $j$ & Description \\
    \midrule
    1 & cumulative probability \\
    2 & total cross section/factor \\
    3 & elastic cross section/factor  \\
    4 & fission cross section/factor \\
    5 & $(n,\gamma)$ cross section/factor \\
    6 & neutron heating number/factor \\
    \bottomrule
  \end{tabular}
  \label{tab:Argumentj}
\end{table}

\begin{ThreePartTable}
  \begin{TableNotes}
  \item[$\dagger$] \label{tn:UNRInterpolationFlag} 
      \begin{description}[font=\ttfamily]
        \item[2] linear-linear interpolation,
        \item[5] log-log interpolation
      \end{description}
  \end{TableNotes}
\begin{BlockTable}{UNR}
  \jxs{23}       & $N$                 & Number of incident energies where there is a probability table. \\
  \jxs{23}+1     & $M$                 & Length of probability table. \\
  \jxs{23}+2     & \var{INT}           & Interpolation parameter between tables.\tnotex{tn:UNRInterpolationFlag} \\
  \jxs{23}+3     & \var{ILF}           & Inelastic competition flag. \\
  \jxs{23}+4     & \var{IOA}           & Other absorption flag. \\
  \jxs{23}+5     & \var{IFF}           & Factors flag. \\
  \jxs{23}+6     & $E(i),i=1,\ldots,N$ & Incident energies. \\
  \jxs{23}+6+$N$ & $P(i,j,k)$          & Probability tables.
  \label{tab:UNRBlock}
\end{BlockTable}
\end{ThreePartTable}

The ordering of the probability table entries, $P(i,j,k)$ is given in \Tableref{tab:PTableOrder}, which begins at $\var{PTABLE}=\jxs{23}+6+N$.
\begin{ThreePartTable}
  \begin{TableNotes}
  \item[$\dagger$] \label{tn:CumulativeProbabilities} The cumulative probabilities are monotonically increasing from an implied (but not included) lower value of zero to the upper value of $P(i,1,k=M)=1.0$.
  \end{TableNotes}
\begin{XSSTable}{Order of probability table elements $P(i,j,k)$}
  \var{PTABLE}         & $\CDF_{1}$              & Cumulative probabilities\tnotex{tn:CumulativeProbabilities}\ for energy $i=1$ \\
  \var{PTABLE}+$M$     & $\sigma_{t,1}$          & Total cross section/factors for energy $i=1$ \\
  \var{PTABLE}+$2M$    & $\sigma_{s,1}$          & Elastic cross section/factors for energy $i=1$ \\
  \var{PTABLE}+$3M$    & $\sigma_{f,1}$          & Fission cross section/factors for energy $i=1$ \\
  \var{PTABLE}+$4M$    & $\sigma_{(n,\gamma),1}$ & $(n,\gamma)$ cross section/factors for energy $i=1$ \\
  \var{PTABLE}+$5M$    & $H_{1}$                 & Heating number/factors for energy $i=1$ \\
  \ldots \\
  \var{PTABLE}         & \multirow{2}{*}{$\CDF_{i}$}              & \multirow{2}{*}{Cumulative probabilities for energy $i$} \\
  \hfill+$(i-1)*6M$    &                         & \\
  \ldots \\
  \var{PTABLE}         & \multirow{2}{*}{$H_{N}$}                 & \multirow{2}{*}{Heating numbers/factors for energy $i=N$} \\
  \hfill+$(N-1)*6M+5M$ &                         & 
  \label{tab:PTableOrder}
\end{XSSTable}
\end{ThreePartTable}
