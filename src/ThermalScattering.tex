%!TEX root = ../ACEFormat.tex
\section[Thermal Scattering]{Thermal Scattering \SaB}\label{sec:ThermalScattering}

Data from thermal \SaB\ tables provide a complete representation of thermal neutron scattering by molecules
and crystalline solids. Cross sections for (coherent and incoherent) elastic and (incoherent) inelastic
scattering are found on the tables. A coupled energy/angle representation is used to describe the spectra
of inelastically scattered neutrons. Angular distributions for elastic scattering are also provided.

Four unique types of data blocks are associated with \SaB\ tables. We now briefly describe each of these four data
block types and reference the sections in which their formats are detailed.

\begin{enumerate}
  \item \textbf{\Block{ITIE}}---contains the energy-dependent incoherent inelastic scattering cross sections.
                                The \Block{ITIE}  always exists. See \Sectionref{sec:ITIEBlock}.
  \item \textbf{\Block{ITCE}} and \textbf{\Block{ITCEI}}---contains the energy-dependent elastic scattering cross sections.
                                The \Block{ITCE} exists if the material has coherent and/or incoherent elastic scattering
                                (\nxs{5}$\neq0$ and \jxs{4}$\neq0$). The \Block{ITCEI} only exists for mixed mode elastic scattering
                                (\nxs{5}$=5$ and \jxs{7}$\neq0$). See \Sectionref{sec:ITCEBlock}.
  \item \textbf{\Block{ITXE}}---contains coupled energy/angle distributions for incoherent inelastic scattering.
                                The \Block{ITXE} always exists. See \Sectionref{sec:ITXEBlock}.
  \item \textbf{\Block{ITCA}} and \textbf{\Block{ITCAI}}---contains angular distributions for elastic scattering.
                                The \Block{ITCA} exists if the material has coherent and/or incoherent elastic scattering
                                (\nxs{5}$\neq0$, \jxs{6}$\neq0$ and \nxs{6}$\neq-1$). The \Block{ITCEI} only exists for mixed mode
                                elastic scattering (\nxs{5}$=5$, \jxs{9}$\neq0$ and \nxs{6}$\neq-1$). See \Sectionref{sec:ITCABlock}.
\end{enumerate}

\subsection{\NXS\ Array}\label{sec:NXSThermalScattering}

\begin{NXSTable}{thermal scattering}
  1  & ---         & Length of second block of data (\XSS\ array)                \\
  2  & \var{IDPNI} & Inelastic scattering mode                                   \\
  3  & \var{NIL}   & Inelastic dimensioning parameter                            \\
  4  & \var{NIEB}  & Number of inelastic exiting energies                        \\
  5  & \var{IDPNC} & Elastic scattering mode (no elastic data=0, incoherent=3, coherent=4, mixed=5) \\
  6  & \var{NCL}   & Elastic dimensioning parameter for the first elastic block  \\
  7  & \var{IFENG} & Secondary energy mode (discrete=0, skewed=1, continuous=2)  \\
  8  & \var{NCLI}  & Elastic dimensioning parameter for the second elastic block \\
     & \ldots      & \\
  16 & ---         &
  \label{tab:NXSThermalScattering}
\end{NXSTable}

\subsection{\JXS\ Array}\label{sec:JXSThermalScattering}

\begin{JXSTable}{thermal scattering}
  1  & \var{ITIE}  & Inelastic energy table               \\
  2  & \var{ITIX}  & Inelastic cross sections             \\
  3  & \var{ITXE}  & Inelastic energy/angle distributions \\
  4  & \var{ITCE}  & Elastic energy table (used for coherent elastic scattering if \\
     &             & \nxs{5}=4 or 5, and used for incoherent elastic scattering if \nxs{5}=3)                 \\
  5  & \var{ITCX}  & Elastic cross sections (used for coherent elastic scattering if \\
     &             & \nxs{5}=4 or 5, and used for incoherent elastic scattering if \nxs{5}=3) \\
  6  & \var{ITCA}  & Elastic angular distributions (used for coherent elastic scattering \\
     &             & if \nxs{5}=4 or 5, and used for incoherent elastic scattering if \nxs{5}=3)        \\
  7  & \var{ITCEI} & Elastic energy table (used for incoherent elastic scattering if \nxs{5}=5)                \\
  8  & \var{ITCXI} & Elastic cross sections (used for incoherent elastic scattering if \nxs{5}=5)              \\
  9  & \var{ITCAI} & Elastic angular distributions (used for incoherent elastic scattering if \nxs{5}=5)       \\
     & \ldots      & \\
  32 & ---         &
  \label{tab:JXSThermalScattering}
\end{JXSTable}

When a single mode of elastic scattering is used (either coherent when \nxs{5}$=4$ or incoherent when \nxs{5}$=3$), then only the first
elastic block will be used. This is the way data was stored in the ACE format prior to the introduction
of mixed mode elastic scattering in thermal scattering evaluations that combines both coherent and incoherent
elastic scattering.

When using mixed mode elastic scattering (both coherent and incoherent elastic scattering are given,
$\nxs{5}=5$), the JXS array will contain an additional set of indices for a second elastic data block. In mixed
mode, the first elastic blocks (pointed to by \jxs{4} to \jxs{6}) are used for the coherent part and the second
elastic blocks (pointed to by \jxs{7} to \jxs{9)} are used for the incoherent part.

\subsection{Format of Individual Data Blocks}
\subsubsection{\textsf{ITIE} Block}\label{sec:ITIEBlock}

The format of the \Block{ITIE} is given in \Tableref{tab:ITIEBlock}. The index at the start of the block is \startblock{ITIE}=\jxs{1}.
Note that \jxs{2}=\jxs{1}+1+$N_{in}$. Linear-linear interpolation is assumed between adjacent energies.

\begin{BlockTable}{ITIE}
  \startblock{ITIE}            & $N_{in}$                           & Number of inelastic energies \\
  \startblock{ITIE}+1          & $E_{in}(l),l=1,\ldots,N_{in}$      & Energies                     \\
  \startblock{ITIE}+1+$N_{in}$ & $\sigma_{in}(l),l=1,\ldots,N_{in}$ & Inelastic cross sections
  \label{tab:ITIEBlock}
\end{BlockTable}

\subsubsection{\textsf{ITXE} Block}\label{sec:ITXEBlock}

The format of the coupled energy/angle distribution for incoherent inelastic scattering is governed by the value
of \nxs{7}. There are three possibilities:
\begin{description}
  \item[$\nxs{7}=0$] equally-likely discrete cosines and energies (\Tableref{tab:ITXEBlock})
  \item[$\nxs{7}=1$] skewed distribution of discrete cosines and energies (\Tableref{tab:ITXEBlock})
  \item[$\nxs{7}=2$] continuous distribution of outgoing energies and equally-likely discrete cosines
                     (\Tableref{tab:ITXEBlockContinuousHeader} and \Tableref{tab:ITXEBlockContinuousData})
\end{description}

The format of the \Block{ITXE} for $\nxs{7}<2$ is given in \Tableref{tab:ITXEBlock}. The index at the start of the
block is \startblock{ITXE}=\jxs{3}. For each incident energy from the \Block{ITIE}, $N'=$\nxs{4} discrete outgoing
energies are given. For each pair of incident and outgoing energies, $N_\mu=$\nxs{3}+1 discrete cosines are given.
The incident inelastic energy grid $E_{in}(l)$ is given in the \Block{ITIE}, and linear-linear interpolation is assumed
between adjacent values of $E_{in}$.

\begin{ThreePartTable}
  \begin{TableNotes}
    \item[$\dagger$] \label{tn:nieb} The number of outgoing energies \var{NIEB} is determined as \nxs{4}.
  \end{TableNotes}
  \begin{XSSTable}{\Block{ITXE} for $\nxs{7}<2$}
  \startblock{ITXE}                       & $E_1^{out}[E_{in}(1)]$                     & First of \var{NIEB}\tnotex{tn:nieb}\ outgoing energies for inelastic scattering at $E_{in}(1)$    \\
  \startblock{ITXE}+1                     & $\mu_l(1\rightarrow 1), l=1,\ldots,N_\mu$  & Discrete cosines for scattering from $E_{in}(1)$ to $E_1^{out}[E_{in}(1)]$    \\
  \startblock{ITXE}+1+$N_\mu$             & $E_2^{out}[E_{in}(1)]$                     & Second of \var{NIEB} outgoing energies for inelastic scattering at $E_{in}(1)$   \\
  \startblock{ITXE}+2+$N_\mu$             & $\mu_l(1\rightarrow 2), l=1,\ldots,N_\mu$  & Discrete cosines for scattering from $E_{in}(1)$ to $E_2^{out}[E_{in}(1)]$    \\
  \multicolumn{1}{c}{\vdots}              & \multicolumn{1}{c}{\vdots}                 & \multicolumn{1}{c}{\vdots}                                                    \\
  \startblock{ITXE}+($N'$-1)(1+$N_\mu$)   & $E_{N'}^{out}[E_{in}(1)]$                  & Last of \var{NIEB} outgoing energies for inelastic scattering at $E_{in}(1)$   \\
  \startblock{ITXE}+($N'$-1)(1+$N_\mu$)+1 & $\mu_l(1\rightarrow N'), l=1,\ldots,N_\mu$ & Discrete cosines for scattering from $E_{in}(1)$ to $E_{N'}^{out}[E_{in}(1)]$ \\
  \midrule
  \multicolumn{3}{l}{(Repeat for all remaining values of $E_{in}$)}
  \label{tab:ITXEBlock}
  \end{XSSTable}
\end{ThreePartTable}

When $\nxs{7}=0$, each of the \nxs{4} discrete outgoing energies for a given incident energy are equally probable. When $\nxs{7}=1$, the
selection of the discrete outgoing energies is skewed such that the first two and last two outgoing energies have a lower probability of
being selected than all other outgoing energies. The first and last energies have a relative probability of 1, the second and second-to-last
energies have a relative probability of 4, and all other energies have a relative probability of 10.

Because the use of discrete outgoing energies and cosines can result in unphysical spikes in the neutron flux spectrum at thermal energies,
some Monte Carlo codes attempt to ``smear'' the outgoing energies and cosines to produce a smoother distribution (that more closely approximates
a continuous distribution with $\nxs{7}=2$).

When $\nxs{7}=2$, the distribution of outgoing energies for each incident energy is continuous in energy and specified by a probability
density function and cumulative distribution function. The format of the \Block{ITXE} in this case is given in \Tableref{tab:ITXEBlockContinuousHeader}
and \Tableref{tab:ITXEBlockContinuousData}. As before, the index at the start of the block is \startblock{ITXE}=\jxs{3}. Unlike in the $\nxs{7} < 2$
cases, the number of outgoing energies for each incident energy is allowed to vary. The number of discrete cosines, $N_\mu=$\nxs{3}-1, remains the same
for each pair of incident and outgoing energies, however.

\begin{ThreePartTable}
  \begin{TableNotes}
    \item[$\dagger$] \label{tn:nin} The number of incoming energies $N_{in}$ for incoherent inelastic scattering is given in the \Block{ITIE}.
  \end{TableNotes}
  \begin{XSSTable}{\Block{ITXE} for $\nxs{7}=2$}
    \startblock{ITXE}          & \var{L}$(l),l=1,\ldots,N_{in}$\tnotex{tn:nin}  & Location in \XSS\ of distribution for incident energy $l$ \\
    \startblock{ITXE}+$N_{in}$ & $N'(l),l=1,\ldots,N_{in}$      & Number of outgoing energies for incident energy $l$
    \label{tab:ITXEBlockContinuousHeader}
  \end{XSSTable}
\end{ThreePartTable}

\begin{XSSTable}{\Block{ITXE} for $\nxs{7}=2$ (continued)}
  \var{L}(1)+1                     & $E_1^{out}[E_{in}(1)]$                      & First of \var{NIEB}\tnotex{tn:nieb}\ outgoing energies for inelastic scattering at $E_{in}(1)$    \\
  \var{L}(1)+2                     & $\mathrm{PDF}_1[E_{in}(1)]$                 & Probability density function value for $E_1^{out}[E_{in}(1)]$ \\
  \var{L}(1)+3                     & $\mathrm{CDF}_1[E_{in}(1)]$                 & Cumulative distribution function value for $E_1^{out}[E_{in}(1)]$ \\
  \var{L}(1)+4                     & $\mu_l(1\rightarrow 1), l=1,\ldots,N_\mu$   & Discrete cosines for scattering from $E_{in}(1)$ to $E_1^{out}[E_{in}(1)]$    \\
  \var{L}(1)+4+$N_\mu$             & $E_2^{out}[E_{in}(1)]$                      & Second of \var{NIEB} outgoing energies for inelastic scattering at $E_{in}(1)$    \\
  \var{L}(1)+5+$N_\mu$             & $\mathrm{PDF}_2[E_{in}(1)]$                 & Probability density function value for $E_2^{out}[E_{in}(1)]$ \\
  \var{L}(1)+6+$N_\mu$             & $\mathrm{CDF}_2[E_{in}(1)]$                 & Cumulative distribution function value for $E_2^{out}[E_{in}(1)]$ \\
  \var{L}(1)+7+$N_\mu$             & $\mu_l(1\rightarrow 2), l=1,\ldots,N_\mu$   & Discrete cosines for scattering from $E_{in}(1)$ to $E_2^{out}[E_{in}(1)]$    \\
  \multicolumn{1}{c}{\vdots}       & \multicolumn{1}{c}{\vdots}                  & \multicolumn{1}{c}{\vdots}                                                    \\
  \var{L}(1)+1+($N'(1)$-1)(3+$N_\mu$) & $E_{N'(1)}^{out}[E_{in}(1)]$                   & Last of $N'(1)$ outgoing energies for inelastic scattering at $E_{in}(1)$    \\
  \var{L}(1)+2+($N'(1)$-1)(3+$N_\mu$) & $\mathrm{PDF}_{N'(1)}[E_{in}(1)]$              & Probability density function value for $E_{N'(1)}^{out}[E_{in}(1)]$ \\
  \var{L}(1)+3+($N'(1)$-1)(3+$N_\mu$) & $\mathrm{CDF}_{N'(1)}[E_{in}(1)]$              & Cumulative distribution function value for $E_{N'(1)}^{out}[E_{in}(1)]$ \\
  \var{L}(1)+4+($N'(1)$-1)(3+$N_\mu$) & $\mu_l(1\rightarrow N'(1)), l=1,\ldots,N_\mu$  & Discrete cosines for scattering from $E_{in}(1)$ to $E_{N'(1)}^{out}[E_{in}(1)]$    \\
  \midrule
  \multicolumn{3}{l}{(Repeat for all remaining values of $E_{in}$)}
  \label{tab:ITXEBlockContinuousData}
\end{XSSTable}

\subsubsection{\textsf{ITCE} Block}\label{sec:ITCEBlock}\label{sec:ITCEIBlock}

The format of the \Block{ITCE} and \Block{ITCEI} is given in \Tableref{tab:ITCEBlock}. The index at the start of the \Block{ITCE}
and \Block{ITCEI} is respectively \startblock{ITCE}=\jxs{4} and \startblock{ITCE}=\jxs{7}.

\begin{BlockTable}{ITCE}
  \startblock{ITCE}            & $N_{el}$                      & Number of elastic energies \\
  \startblock{ITCE}+1          & $E_{el}(l),l=1,\ldots,N_{el}$ & Energies                   \\
  \startblock{ITCE}+1+$N_{el}$ & $P(l),l=1,\ldots,N_{el}$      & (See below)
  \label{tab:ITCEBlock}
\end{BlockTable}

For incoherent elastic scattering (stored in \Block{ITCE} if $\nxs{5}=4$ and stored in \Block{ITCEI} if $\nxs{5}=5$),
\begin{equation}
  P(l) = \sigma_{el}(E_{el}(l))
\end{equation}
with linear-linear interpolation between points. For coherent elastic scattering (stored in \Block{ITCE} if \nxs{5}$=4 \text{or} 5$),
\begin{equation}
  P(l) = E\cdot\sigma_{el}(E), \qquad E_{el}(l) \le E < E_{el}(l+1).
  \label{eq:CoherentP}
\end{equation}
In this case, the energies $E_{el}(l)$ correspond to Bragg edges, and between two energies the cross section
is determined by inverting \Equationref{eq:CoherentP}:
\begin{equation}
  \sigma_{el}(l) = \frac{P(l)}{E}, \qquad E_{el}(l) \le E < E_{el}(l+1).
\end{equation}
Also note that $\sigma_{el}(E)=0$ below $E_{el}(1)$. However, above $E_{el}(N_{el})$, $\sigma_{el}(E) = P(N_{el})/E$.

\subsubsection{\textsf{ITCA} Block}\label{sec:ITCABlock}\label{sec:ITCAIBlock}

The format of the \Block{ITCA} and \Block{ITCAI} is given in \Tableref{tab:ITCABlock}. The index at the start of the \Block{ITCA}
and \Block{ITCAI} is respectively \startblock{ITCA}=\jxs{6} and \startblock{ITCA}=\jxs{9}. For each incident energy from the
\Block{ITCE} and \Block{ITCEI} respectively, $N_\mu=$\nxs{6}+1 and $N_\mu=$\nxs{8}+1 discrete cosines are given.

\begin{BlockTable}{ITCA}
  \startblock{ITCA}                     & $\mu_l[E_{el}(1)], l=1,\ldots,N_\mu$      & Discrete cosines for elastic scattering at $E_{el}(1)$ \\
  \startblock{ITCA}+$N_\mu$             & $\mu_l[E_{el}(2)], l=1,\ldots,N_\mu$      & Discrete cosines for elastic scattering at $E_{el}(2)$ \\
  \multicolumn{1}{c}{\vdots}            & \multicolumn{1}{c}{\vdots}                & \multicolumn{1}{c}{\vdots}                             \\
  \startblock{ITCA}+($N_{el}$-1)$N_\mu$ & $\mu_l[E_{el}(N_{el})], l=1,\ldots,N_\mu$ & Discrete cosines for elastic scattering at $E_{el}(N_{el})$
  \label{tab:ITCABlock}
\end{BlockTable}

The incident elastic energy grid $E_{el}(l)$ is given in the \Block{ITCE} or \Block{ITCEI}. Linear-linear interpolation is assumed between adjacent values of $E_{el}$.
