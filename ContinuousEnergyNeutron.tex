%!TEX root = ACEFormat.tex
\section{Continuous-Energy and Discrete Neutron Transport Tables}\label{sec:ContinuousEnergyNeutron}
The format of individual blocks found on neutron transport tables is identical for continuous-energy and discrete-reaction \ACE\ Tables; the format for both are described in this section. The blocks of data are:
\begin{enumerate}
  \item \textbf{\Block{ESZ}}---contains the main energy gid for the Table and the total, absorption, and elastic cross sections as wella s the average heating numbers. The \Block{ESZ} block always exists. See \Sectionref{sec:ESZBlock}.
  \item \textbf{\Block{NU}}---contains prompt, delayed and/or total $\overline{\nu}$ as a function of incident neutron energy. The \Block{NU} exists only for fissionable isotopes; that is, if \jxs{2}$\neq0$. See \Sectionref{sec:NUBlock}.
  \item \textbf{\Block{MTR}}---contains a list of ENDF MT numbers for all neutron reactions other than elastic scattering. The \Block{MTR} exists for all isotopes that have reactions other than elastic scattering; that is, all isotopes with \nxs{4}$\neq0$. See \Sectionref{sec:MTRBlock}.
  \item \textbf{\Block{LQR}}---contains a list of kinematic $Q$-values for all neutron reactions other than elastic scattering. The \Block{LTR} exists if \nxs{4}$\neq0$. See \Sectionref{sec:LQRBlock}.
  \item \textbf{\Block{TYR}}---contains information about the type of reaction for all neutron reactions other than elastic scattering. Information for each reaction includes the number of secondary neutrons and whether secondary neutron angular distributions are in the laboratory or \CM system. The \Block{TYR} exists if \nxs{4}$\neq0$. See \Sectionref{sec:TYRBlock}.
  \item \textbf{\Block{LSIG}}---contains a list of cross section locators for all neutron reacitons other than elastic scattering. The \Block{LSIG} exists if \nxs{4}$\neq0$. See \Sectionref{sec:LSIGBlock}
  \item \textbf{\Block{SIG}}---contains cross sections for all reactions other than elastic scattering. The \Block{SIG} exists if \nxs{4}$\neq0$. See \Sectionref{sec:SIGBlock}.
  \item \textbf{\Block{LAND}}---contains a list of angular-distribution locators for all reactions producing secondary neutrons. The \Block{LAND} always exists. See \Sectionref{sec:LANDBlock}.
  \item \textbf{\Block{AND}}---contains list angular distributions for all reactions producing secondary neutrons. The \Block{AND} always exists. See \Sectionref{sec:ANDBlock}.
  \item \textbf{\Block{LDLW}}---contains a list of energy distributions for all reactions producing secondary neutrons except for elastic scattering. The \Block{LDLW} exists if \nxs{5}$\neq0$. See \Sectionref{sec:LDLWBlock}.
  \item \textbf{\Block{DLW}}---contains energy distributions for all reactions producing secondary neutrons except for elastic scattering. The \Block{DLW} exists if \nxs{5}$\neq0$. See \Sectionref{sec:DLWBlock}.
  \item \textbf{\Block{GPD}}---contains the total photon production cross section tabulated on the ESZ energy grid and a $30\times$ matrix of secondary photon energies. The \Block{GPD} exists only for those older evaluations that provide coupled neutron/photon information; that is, if \jxs{12}$\neq0$. See \Sectionref{sec:GPDBlock}.
  \item \textbf{\Block{MTRP}}---contains a list of MT numbers for all photon production reactions. The term ``photon production reaction'' is used for any information describing a specific neutron-in, photon-out reaction. The \Block{MTR} exists if \nxs{6}$\neq6$. See \Sectionref{sec:MTRBlock}.
  \item \textbf{\Block{LSIGP}}---contains a list of cross section locators for all photon production reactions. The \Block{LSIGP} exists if \nxs{6}$\neq0$. See \Sectionref{sec:LSIGBlock}.
  \item \textbf{\Block{SIGP}}---contains cross sections for all photon production reactions. The \Block{SIGP} exists if \nxs{6}$\neq0$. See \Sectionref{sec:SIGPBlock}.
  \item \textbf{\Block{LANDP}}---contains a list of angular-distribution locators for all photon production reactions. The \Block{LANDP} exist if \nxs{6}$\neq0$. See \Sectionref{sec:LANDPBlock}
  \item \textbf{\Block{ANDP}}---contains photon angular distributions for all photon production reactions. The \Block{ANDP} exists if \nxs{6}$\neq0$. See \Sectionref{sec:ANDPBlock}.
  \item \textbf{\Block{LDLWP}}---contains a list of energy-distribution locators for all photon production reactions. The \Block{LDLWP} exists if \nxs{6}$\neq0$. See \Sectionref{sec:LDLWPBlock}.
  \item \textbf{\Block{DLWP}}---contains photon energy distributions for all photon production reactions. The \Block{DLWP} exists if \nxs{6}$\neq0$. See \Sectionref{sec:DLWPBlock}.
  \item \textbf{\Block{YP}}---contains a list of MT identifiers of neutron reaction cross sections required as photon production yield multipliers. The \Block{YP} exists if \nxs{6}$\neq0$. See \Sectionref{sec:YPBlock}.
  \item \textbf{\Block{FIS}}---contains the total fission cross section tabulated on the ESZ energy grid. The \Block{FIS} exists if \jxs{21}$\neq0$. See \Sectionref{sec:FISBlock}.
  \item \textbf{\Block{UNR}}---contains the unresolved resonance range probability tables. The \Block{UNR} exists if \jxs{23}$\neq0$. See \Sectionref{sec:UNRBlock}.
\end{enumerate}

\subsection{\NXS\ Array}\label{sec:NXSContinuousEnergyNeutron}

\begin{NXSTable}{continuous-energy neutron}
    1        & ---    & Length of second block of data (\XSS\ array) \\
    2        & ZA     & $1000*Z+A$ \\
    3        & NES    & Number of energies \\
    4        & NTR    & Number of reactions excluding elastic scattering \\
    5        & NR     & Number of reactions having secondary neutrons excluding elastic scattering \\
    6        & NTRP   & Number of photon production reactions \\
             & \ldots & \\
    8        & NPCR   & Number of delayed neutron precurser families \\
             & \ldots & \\
    15       & NT     & Number of PIKMT reaction \\
    16       & ---    & 0=normal photon production \\
             &        & -1=do not produce photons
  \label{tab:NXSContinuousEnergyNeutron}
\end{NXSTable}

\todo[inline]{Does NXS[15] apply to every type of data, or just fast tables?}

\subsection{\JXS\ Array}\label{sec:JXSContinuousEnergyNeutron}
\begin{JXSTable}{continuous-energy neutron}
    1        & ESZ    & Energy table \\
    2        & NU     & Fission $\nu$ data \\
    3        & MTR    & \texttt{MT} array \\
    4        & LQR    & $Q$-value array \\
    5        & TYR    & Reaction type array \\
    6        & LSIG   & Table of cross section locators \\
    7        & SIG    & Cross sections \\
    8        & LAND   & Table of angular distribution locators \\
    9        & AND    & Angular distributions \\
    10       & LDLW   & Table of energy distribution locators \\
    11       & DLW    & Energy distributions \\
    12       & GPD    & Photon production data \\
    13       & MTRP   & Photon production \texttt{MT} array \\
    14       & LSIGP  & Table of photon production cross section locators \\
    15       & SIGP   & Photon production cross sections \\
    16       & LANDP  & Table of photon production angular distribution locators \\
    17       & ANDP   & Photon production angular distributions \\
    18       & LDLWP  & Table of photon production energy distribution locators \\
    19       & DLWP   & Photon production energy distributions \\
    20       & YP     & Table of yield multipliers \\
    21       & FIS    & Total fission cross section \\
    22       & END    & Last word of this table \\
    23       & LUNR   & Probability tables \\
    24       & DNU    & Delayed $\overline{\nu}$ data \\
    25       & BDD    & Basic delayed data ($\lambda$'s, probabilities) \\
    26       & DNEDL  & Table of energy distribution locators \\
    27       & DNED   & Energy distributions \\
             & \ldots & \\
    32       & ---    &
  \label{tab:JXSContinuousEnergyNeutron}
\end{JXSTable}

\subsection{Format of Individual Data Blocks}
\subsubsection{\Block{ESZ}}\label{sec:ESZBlock}
The format of the \Block{ESZ} is given in \Tableref{tab:ESZBlock}.
\begin{ThreePartTable}
  \begin{BlockTable}{ESZ}
    \startblock{ESZ}            & $E(l), l=1,\ldots, N_{E}$           & Energies \\
    \startblock{ESZ} + $N_{E}$  & $\sigma_{t}(l), l=1,\ldots, N_{E}$  & Total cross section \\
    \startblock{ESZ} + $2N_{E}$ & $\sigma_{s}(l), l=1,\ldots, N_{E}$  & Total absorption cross section \\
    \startblock{ESZ} + $3N_{E}$ & $\sigma_{el}(l), l=1,\ldots, N_{E}$ & Elastic cross section \\
    \startblock{ESZ} + $4N_{E}$ & $H_{el}(l), l=1,\ldots, N_{E}$      & Average Heating numbers
    \label{tab:ESZBlock}
  \end{BlockTable}
  \begin{tablenotes}
    \note \startblock{ESZ} is index of the \XSS\ array where the \Block{ESZ} starts, \jxs{1},  and $N_{E}$ is the number of energy energy points, \nxs{3}.
  \end{tablenotes}
\end{ThreePartTable}

\subsubsection{\Block{NU}}\label{sec:NUBlock}
There are four possibilities for the \Block{NU}:
\begin{enumerate}
  \item No \Block{NU}. This happens when \jxs{2}=0.
  \item Either prompt or total \nubar\ is given (but not both). The \aceArray{NU} array begins at location \xss{KNU} where \var{KNU}=\jxs{2}.
  \item Both prompt and total \nubar\ are given. The prompt \aceArray{NU} array begins at \xss{KNU} where \var{KNU}=\jxs{2}; the total \aceArray{NU} array begins at \xss{KNU} where {\sffamily \var{KNU} = \jxs{2} + ABS(\xss{\jxs{2}})+1}
  \item Delayed \nubar\ is given. The delayed \nubar\ array begins at \xss{KNU} where\\ \var{KNU}=\jxs{24}. Delayed \nubar\ must be given in form b described below.
\end{enumerate}

The format of the \Block{NU} has two forms (if it exists); polynomial (see \Tableref{tab:NUBlockPolynomial}) and tabulated (see \Tableref{tab:NUBlockTabulated}). The format is specified by the \var{LNU} flag located in the \XSS\ array at index \var{KNU} where \var{KNU} is defined above.
\begin{BlockTable}[---Polynomial function form]{NU}
  \var{KNU}   & \var{LNU}=1                     & Polynomial function flag \\
  \var{KNU}+1 & $N_{C}$                   & Number of coefficients \\
  \var{KNU}+2 & $C(l), l=1,\ldots, N_{C}$ & Coefficients
  \label{tab:NUBlockPolynomial}
\end{BlockTable}
When using the polynomial function form of the \aceArray{NU} array, \nubar\ is reconstructed as
\begin{equation}
  \nubar(E) = \sum_{l=1}^{N_{C}} C(l)E^{l-1},
  \label{eq:nubarPolynomialReconstruction}
\end{equation}
where the energy, $E$, is given in \si{\MeV}.

\begin{ThreePartTable}
  \begin{TableNotes}
  \item[$\dagger$] \label{tn:scheme} If $N_{R}=0$, \var{NBT} and \var{INT} are omitted and linear-linear interpolation is assumed.
  \end{TableNotes}
  \begin{BlockTable}[---Tabulated form]{NU}
    \var{KNU}                   & \var{LNU}=2                  & Tabulated data flag \\
    \var{KNU}+1                 & $N_{R}$                      & Number of interpolation regions \\
    \var{KNU}+2                 & \var{NBT}$(l), l=1,\ldots,N_{R}$   & ENDF interpolation parameters \\
    \var{KNU}+2+$N_{R}$         & \var{INT}$(l), l=1,\ldots,N_{R}$   & ENDF interpolation scheme\tnotex{tn:scheme} \\
    \var{KNU}+2+$2N_{R}$       & $N_{E}$                      & Number of energies \\
    \var{KNU}+3+$2N_{R}$       & $E(l),l=1,\ldots,N_{E}$      & Tabulated energy points \\
    \var{KNU}+3+$2N_{R}+N_{E}$ & $\nubar(l),l=1,\ldots,N_{E}$ & Tabulated \nubar\ values
    \label{tab:NUBlockTabulated}
  \end{BlockTable}
\end{ThreePartTable}

If delayed neutron data exist (when \jxs{24}>0), the precursor distribution format is given as in \Tableref{tab:DelayedPrecursorDistribution}. The decay constant for the first group \var{DEC}$_{1}$ is given at \xss{\jxs{25}}. The precursor distribution immediately follows as described in \Tableref{tab:DelayedPrecursorDistribution}. The indices (locators) of the \XSS\ array where each precursor distribution begins (\startblock{DNU}) can found using the format described in \Sectionref{sec:LDLWPBlock} and \Sectionref{sec:DLWPBlock}, where \var{LED}=\jxs{26} and \var{NMT}=\nxs{8}.
\begin{ThreePartTable}
  \begin{TableNotes}
    \item[$\dagger$] \label{tn:schemeDelayedPrecursors} If $N_{R}=0$, \var{NBT} and \var{INT} are omitted and linear-linear interpolation is assumed.
  \end{TableNotes}
  \begin{XSSTable}{Delayed \nubar\ precursor distribution.}
    \startblock{DNU}                   & \var{DEC}$_{i}$                  & Decay constant for the $i$-th group \\
    \startblock{DNU}+1                 & $N_{R}$                          & Number of interpolation regions \\
    \startblock{KNU}+2                 & \var{NBT}$(l), l=1,\ldots,N_{R}$ & ENDF interpolation parameters \\
    \startblock{KNU}+2+$N_{R}$         & \var{INT}$(l), l=1,\ldots,N_{R}$ & ENDF interpolation scheme\tnotex{tn:schemeDelayedPrecursors} \\
    \startblock{DNU}+2+$2N_{R}$       & $N_{E}$                          & Number of energies \\
    \startblock{DNU}+3+$2N_{R}$       & $E(l),l=1,\ldots,N_{E}$          & Tabulated energy points \\
    \startblock{DNU}+3+$2N_{R}+N_{E}$ & $P(l),l=1,\ldots,N_{E}$          & Corresponding probabilities
    \label{tab:DelayedPrecursorDistribution}
  \end{XSSTable}
  \begin{tablenotes}
    \note \startblock{DNU} is the index of the \XSS\ array where the delayed \nubar\ precursor distribution begins; the first one is at \startblock{DNU}=\jxs{25}.
  \end{tablenotes}
\end{ThreePartTable}

\subsubsection{\Block{MTR \textnormal{\&} MTRP}s}\label{sec:MTRBlock}
The format of the \Block{MTR} (for incident neutron reactions) and \Block{MTRP} (for photon production reactions) is given in \Tableref{tab:MTRBlock}. The starting index depends on whether it is the \Block{MTR} or \Block{MTRP} and are given in \Tableref{tab:LMT_NMT}.
\begin{table}[h!] \centering
  \begin{tabular}[h]{lll}
    \toprule
    Block & \var{LMT} & \var{NMT} \\
    \midrule
    \var{MTR} & \jxs{3} & \nxs{4} \\
    \var{MTRP} & \jxs{13} & \nxs{6} \\
    \bottomrule
  \end{tabular}
  \caption{\var{LMT} and \var{NMT} values for the \Block{MTR} and \Block{MTR}.}
  \label{tab:LMT_NMT}
\end{table}

\begin{BlockTable}{MTR \textnormal{\&} MTRP}
  \var{LMT} & \MT$_{1}$ & First ENDF Reaction available \\
  \var{LMT}+1 & \MT$_{2}$ & Second sENDF Reaction available \\
  \ldots \\
  \var{LMT}+\var{NMT}+1 & \MT$_{\var{NMT}}$ & Last ENDF reaction available
  \label{tab:MTRBlock}
\end{BlockTable}

For the \Block{MTR}, \MT$_{1},\ldots,\MT_{\var{NMT}}$ are standard ENDF \MT numbers; that is, \MT=16=$(n,2n)$; \MT=17=$(n,3n)$; etc. For a complete listing of \MT\ numbers, see \cite[Appendix B]{Trkov:2011ENDF--0}.

For the \Block{MTRP}, the \MT\ numbers are somewhat arbitrary. To understand the scheme used for numbering the photon production \MT s, it is necessary to realize that in the ENDF format, more than one photon can be produced by a particular neutron reaction that is itself specified by a single \MT. Each of these photons is produced with an individual energy-dependent cross section. For example, \MT 102 (radiatiive capture) might be responsible for 40 photons, each with its own cross section, angular distribution, and energy distribution. We need 40 photon \MT s to represent the data; the \MT s are numbered \textsf{1002001}, \textsf{1002002}, \ldots, \textsf{1002040}. Therefore, if ENDF \MT\ $N$ is responsible for $M$ photons, we shall number the photon \MT s \textsf{1000*$N$+1}, \textsf{1000*$N$+2}, \ldots, \textsf{1000*$N$+$M$}.

\subsubsection{\Block{LQR}}\label{sec:LQRBlock}
The format of the \Block{LQR}, containing the reaction-specific $Q$-values, is given in \Tableref{tab:LQRBlock}. The index at the start of the \Block{LQR}, \startblock{LQR}=\jxs{4}. The number of reactions, \var{NMT}, is the same through the \ACE\ Table, \var{NMT}=\nxs{4}.
\begin{BlockTable}{LQR}
  \startblock{LQR}           & $Q_{1}$         & $Q$-value for reaction \MT$_{1}$ \\
  \startblock{LQR}+1         & $Q_{2}$         & $Q$-value for reaction \MT$_{2}$ \\
  \ldots \\
  \startblock{LQR}+\var{NMT}-1 & $Q_{\var{NMT}}$ & $Q$-value for reaction \MT$_{\var{NMT}}$
  \label{tab:LQRBlock}
\end{BlockTable}

\subsubsection{\Block{TYR}}\label{sec:TYRBlock}
The format of the \Block{TYR} is given in \Tableref{tab:TYRBlock}. The index at the start of the \Block{TYR}, \startblock{TYR}=\jxs{5}. The number of reactions, \var{NMT}, is the same through the \ACE\ Table, \var{NMT}=\nxs{4}.

\begin{BlockTable}{TYR}
  \startblock{TYR}             & \var{TY}$_{1}$         & Neutron release for reaction \MT$_{1}$ \\
  \startblock{TYR}+1           & \var{TY}$_{2}$         & Neutron release for reaction \MT$_{2}$ \\
  \ldots \\
  \startblock{TYR}+\var{NMT}-1 & \var{TY}$_{\var{NMT}}$ & Neutron release for reaction \MT$_{\var{NMT}}$
  \label{tab:TYRBlock}
\end{BlockTable}
The possible values of \var{TY} are $\pm 1$, $\pm 2$, $\pm 3$, $\pm 4$, $\pm 19$, 0, and integers greater than 100 in absolute value; the sign indicates the system for scattering: negative=\CM, positive=\LAB. Thus if \var{TY}$_{i}$=+3, three neutrons are released for reaction \MT$_{i}$ and the data on the cross section tables used to determine the exiting neutrons' angles are given in the \LAB\ frame of reference. \var{TY}=19 indicates fission. The number of secondary neutrons released is determined from the fission \nubar\ data found in the \Block{NU}. \var{TY}$_{i}$=0 indicates absorption (ENDF reactions \MT>100); no neutrons are released. $\|\var{TY}_{i}\|>100$ signifies reactions other than fission that have energy-dependent neutron multiplicities. The number of secondary neutrons released is determined from the yield data found in the \Block{DLW}. The \MT$_{i}$s are given in the \Block{MTR}.

\subsubsection{\Block{LSIG \textnormal{\&} LSIGP}}\label{sec:LSIGBlock}
The \Block{LSIG} and \Block{LSIGP} gives the locators for cross section array for each reaction \MT. A locator is a \emph{relative} index in the \XSS\ array where some piece of data. In this case, the data is the cross section values. The format of the \Block{LSIG} (for incident neutron cross sections) and \Block{LSIGP} (for photon production cross sections) is given in \Tableref{tab:LSIGBlock}. The format for the incident neutron cross section arrays is given in \Sectionref{sec:SIGBlock}. The format for the photon production cross sections is given in \Sectionref{sec:SIGPBlock}.

All locators are relative to \jxs{7} for the \Block{LSIG} or \jxs{15} for the \Block{LSIGP}. That is, \var{LXS}=\jxs{7} for the \Block{LSIG} and \var{LXS}=\jxs{15} for the \Block{LSIGP}. So the actual cross section data begins at the index \var{LOCA}+\var{LXS}. The \MT s are given in the \Block{MTR} and the \Block{MTRP} for the \Block{LSIG} and the \Block{LSIGP} respectively. \var{LOCA}$_{i}$ must be monotonically increasing.
\begin{BlockTable}{LSIG \textnormal{\&} LSIGP}
  \var{LXS}             & \var{LOCA}$_{1}=1$       & Location of cross sections for reaction \MT$_{1}$ \\
  \var{LXS}+1           & \var{LOCA}$_{2}$         & Location of cross sections for reaction \MT$_{2}$ \\
  \ldots \\
  \var{LXS}+\var{NMT}-1 & \var{LOCA}$_{\var{NMT}}$ & Location of cross sections for reaction \MT$_{\var{NMT}}$
  \label{tab:LSIGBlock}
\end{BlockTable}

\subsubsection{\Block{SIG}}\label{sec:SIGBlock}
The \Block{SIG} contains the incident neutron cross section data. (The photon production cross section is in the \Block{SIGP}.) The format of the \Block{SIG} is given in \Tableref{tab:SIGBlock}. The cross section data begins at the index specified by the locator from the \Block{LSIG}; the format for which is given in \Tableref{tab:CrossSectionArray}.
\begin{ThreePartTable}
  \begin{LOCTable}{\Block{SIG}}
    \var{LXS}+\var{LOCA}$_{1}-1$         & Cross section array for reaction \MT$_{1}$ \\
    \var{LXS}+\var{LOCA}$_{2}-1$         & Cross section array for reaction \MT$_{2}$ \\
    \ldots \\
    \var{LXS}+\var{LOCA}$_{\var{NMT}}-1$ & Cross section array for reaction \MT$_{\var{NMT}}$
    \label{tab:SIGBlock}
  \end{LOCTable}
  \begin{tablenotes}
    \note The number of cross section arrays \var{NMT}=\nxs{4}.
  \end{tablenotes}
\end{ThreePartTable}

The \var{LOCA}$_{i}$ values are given in the \Block{LSIG} and are all relative to \jxs{7}. The energy grid, $E(K)$ is given in the \Block{ESZ}. The energy grid index $IE_{i}$ corresponds to the first energy in the grid at which a cross section is given. The \MT$_{i}$s are defined in the \Block{MTR}.
\begin{XSSTable}{Cross section array for the $i$-th reaction.}
\var{LXS} + \var{LOCA}$_{i}$-1                  & $IE_{i}$                          & Energy grid index for reaction \MT$_{i}$ \\
\var{LXS} + \var{LOCA}$_{i}$                    & $NE_{i}$                          & Number of consecutive entries for \MT$_{i}$ \\
\multirow{2}{*}{\var{LXS} + \var{LOCA}$_{i}$+1} & $\sigma_{i}[E(K)]$ for            & \multirow{2}{*}{Cross section for reaction \MT$_{i}$} \\
                                                & $K=IE_{i},\ldots,IE_{i}+NE_{i}-1$ & 
  \label{tab:CrossSectionArray}
\end{XSSTable}

\subsubsection{\Block{LAND}}\label{sec:LANDBlock}
The \Block{LAND} contains locators for the angular distributions for all reactions producing secondary neutrons. The \Block{LAND} always exists and begins at \startblock{LAND}=\jxs{8}. All locators (\var{LOCB}) are relative \jxs{9}; that is, the angular distribution begins at $\jxs{9}+\var{LOCB}_{i}$. The \var{LOCB}$_{i}$ locators must be monotonically increasing. The format of the \Block{LAND} is given in \Tableref{tab:LANDBlock}.
\begin{ThreePartTable}
\begin{BlockTable}{LAND}
  \startblock{LAND} & \var{LOCB}$_{1}=1$ & Location of angular distribution data for elastic scattering reaction \\
  \startblock{LAND}+1 & \var{LOCB}$_{2}$ & Location of angular distribution data for reaction \MT$_{1}$ \\
  \ldots \\
  \startblock{LAND}+\var{NMT} & \var{LOCB}$_{\var{NMT}}$ & Location of angular distribution data for reaction \MT$_{\var{NMT}}$
  \label{tab:LANDBlock}
\end{BlockTable}
\begin{tablenotes}
  \note \startblock{LAND}=\jxs{8} and \var{NMT}=\nxs{5} is the number of reactions (excluding elastic scattering).
\end{tablenotes}
\end{ThreePartTable}

\subsubsection{\Block{AND}}\label{sec:ANDBlock}
The \Block{AND} contains angular distribution data for all reactions that produce secondary neutrons. The format of the \Block{AND} is given in \Tableref{tab:ANDBlock}. The angular distribution data begins at the index specified by the locator \var{LOCB} from the \Block{LAND}. If \var{LOCB}$_{i}=0$ (given in the \Block{LAND}), no angular distribution data are given for reaction $i$ and isotropic scattering is assumed in either the \LAB\ or \CM\ system. The choice of \LAB\ or \CM\ system depends upon the value for reaction $i$ in the \Block{TYR}. If \var{LOCB}$_{i}=-1$ no angular distribution data are given for reaction $i$ in the \Block{AND}. The angular distribution data are specified through \law{44} in the \Block{DLW}. 

\begin{ThreePartTable}
\begin{LOCTable}{\Block{AND}}
  \jxs{9}+\var{LOCB}$_{1}-1$         & Angular distribution array for elastic scattering \\
  \jxs{9}+\var{LOCB}$_{2}-1$         & Angular distribution array for reaction \MT$_{1}$ \\
  \jxs{9}+\var{LOCB}$_{\var{NMT}}-1$ & Angular distribution array for reaction \MT$_{\var{NMT}}$
  \label{tab:ANDBlock}
\end{LOCTable}
\begin{tablenotes}
  \note The format for the angular distribution of the $i$-th array is given in \Tableref{tab:AngularDistributionArray}.
\end{tablenotes}
\end{ThreePartTable}

\begin{XSSTable}{Angular distribution array for the $i$-th reaction.}
  \jxs{9}+\var{LOCB}$_{i}-1$     & $N_{E}$                     & Number of energies at which angular distributions are tabulated.  \\
  \jxs{9}+\var{LOCB}$_{i}$       & $E(l),l=1,\ldots,N_{E}$     & Energy grid \\
  \jxs{9}+\var{LOCB}$_{i}+N_{E}$ & $L_{C}(l),l=1,\ldots,N_{E}$ & Location of tables associated with $E(l)$
  \label{tab:AngularDistributionArray}
\end{XSSTable}

The angular distribution arrays (\Tableref{tab:AngularDistributionArray}) contains additional locators, $L_{C}$; the sign of these locators is a flag:
\begin{itemize}
  \item if $L_{C}(l)>0$, then $L_{C}(l)$ points to a 32 equiprobable bin distribution (see \Tableref{tab:32EquiprobableBinDistribution});
  \item if $L_{C}(l)<0$, then $L_{C}(l)$ points to a tabulated angular distribution (see \Tableref{tab:TabulatedAngularDistribution});
  \item if $L_{C}(l)=0$, then distribution is isotropic and no further data is needed.
\end{itemize}

\begin{XSSTable}{Format for the 32 equiprobable bin distribution.}
  \multirow{2}{*}{\jxs{9}+$|L_{C}(l)|-1$} & $P(1,K)$        & 32 equiprobable cosine bins for scattering  \\
                                          & $K=1,\ldots,33$ & at energy $E(1)$.
  \label{tab:32EquiprobableBinDistribution}
\end{XSSTable}

\begin{ThreePartTable}
  \begin{TableNotes}
    \item[$\dagger$] \label{tn:InterpolationFlag}
      \begin{description}[font=\ttfamily]
        \item[0] histogram interpolation
        \item[1] linear-linear interpolation
      \end{description}
  \end{TableNotes}
\begin{XSSTable}{Format for the tabulated angular distribution.}
  \var{LDAT}$_{l}+1$ & \var{JJ}                                & Interpolation flag\tnotex{tn:InterpolationFlag} \\
  \var{LDAT}$_{l}+2$ & $N_{P}$                                 & Number of points in the distribution \\
  \var{LDAT}$_{l}+3$ & $CS_{\mathrm{out}}(j),j=1,\ldots,N_{P}$ & Cosine scattering angular grid \\
  \var{LDAT}$_{l}+4$ & $\PDF(j),j=1,\ldots,N_{P}$              & Probability density function \\
  \var{LDAT}$_{l}+5$ & $\CDF(j),j=1,\ldots,N_{P}$              & Cumulative density function
  \label{tab:TabulatedAngularDistribution}
\end{XSSTable}
\begin{tablenotes}
  \note \var{LDAT}$_{l}=\jxs{9}+|L_{C}(l)|-1$
\end{tablenotes}
\end{ThreePartTable}



\subsubsection{\Block{LDLW}}\label{sec:LDLWBlock}
The format of the \Block{LDLW} is given in \Tableref{tab:LDLWBlock}.
\begin{BlockTable}{LDLW}
  \label{tab:LDLWBlock}
\end{BlockTable}

\subsubsection{\Block{DLW}}\label{sec:DLWBlock}
The format of the \Block{DLW} is given in \Tableref{tab:DLWBlock}.
\begin{BlockTable}{DLW}
  \label{tab:DLWBlock}
\end{BlockTable}

\subsubsection{\Block{GPD}}\label{sec:GPDBlock}
The format of the \Block{GPD} is given in \Tableref{tab:GPDBlock}.
\begin{BlockTable}{GPD}
  \label{tab:GPDBlock}
\end{BlockTable}

\subsubsection{\Block{MTRP}}\label{sec:MTRPBlock}
The format of the \Block{MTRP} is given in \Tableref{tab:MTRPBlock}.
\begin{BlockTable}{MTRP}
  \label{tab:MTRPBlock}
\end{BlockTable}

\subsubsection{\Block{SIGP}}\label{sec:SIGPBlock}
The format of the \Block{SIGP} is given in \Tableref{tab:SIGPBlock}.
\begin{BlockTable}{SIGP}
  \label{tab:SIGPBlock}
\end{BlockTable}

\subsubsection{\Block{LANDP}}\label{sec:LANDPBlock}
The format of the \Block{LANDP} is given in \Tableref{tab:LANDPBlock}.
\begin{BlockTable}{LANDP}
  \label{tab:LANDPBlock}
\end{BlockTable}

\subsubsection{\Block{ANDP}}\label{sec:ANDPBlock}
The format of the \Block{ANDP} is given in \Tableref{tab:ANDPBlock}.
\begin{BlockTable}{ANDP}
  \label{tab:ANDPBlock}
\end{BlockTable}

\subsubsection{\Block{LDLWP}}\label{sec:LDLWPBlock}
The format of the \Block{LDLWP} is given in \Tableref{tab:LDLWPBlock}.
\begin{BlockTable}{LDLWP}
  \label{tab:LDLWPBlock}
\end{BlockTable}

\subsubsection{\Block{DLWP}}\label{sec:DLWPBlock}
The format of the \Block{DLWP} is given in \Tableref{tab:DLWPBlock}.
\begin{BlockTable}{DLWP}
  \label{tab:DLWPBlock}
\end{BlockTable}

\subsubsection{\Block{YP}}\label{sec:YPBlock}
The format of the \Block{YP} is given in \Tableref{tab:YPBlock}.
\begin{BlockTable}{YP}
  \label{tab:YPBlock}
\end{BlockTable}

\subsubsection{\Block{FIS}}\label{sec:FISBlock}
The format of the \Block{FIS} is given in \Tableref{tab:FISBlock}.
\begin{BlockTable}{FIS}
  \label{tab:FISBlock}
\end{BlockTable}

\subsubsection{\Block{UNR}}\label{sec:UNRBlock}
The format of the \Block{UNR} is given in \Tableref{tab:UNRBlock}.
\begin{BlockTable}{UNR}
  \label{tab:UNRBlock}
\end{BlockTable}
